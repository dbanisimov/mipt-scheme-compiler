\documentclass[16pt,pdf,unicode]{beamer}

\usepackage[utf8]{inputenc}
\usepackage[english,russian]{babel}
\usepackage{listings}
\usepackage{tikz}
\usepackage{color}
\usepackage{geometry}

\usetikzlibrary{positioning}
\usetikzlibrary{automata}
\usetikzlibrary{chains}
\usetikzlibrary{shapes,arrows}
\usetikzlibrary{calc} % for manimulation of coordinates

\usetheme{Boadilla}
\setbeamercovered{transparent}
\lstset{ language=Lisp }

\title[Лексический анализ]{Теория и практика компилляции программ: лексический анализ}
\author[А.Ю. Заостровных]{Заостровных Арсений Юрьевич}
\institute[МФТИ]{Московский физико-технический университет\\
	{\tiny государственный университет}\\}
\date[2012]{МФТИ, 27 октября 2012 года}

\begin{document}

%\iffalse

\begin{frame}[plain]
\maketitle
\end{frame}

\begin{frame}
  \frametitle{Положение лексического анализа}

\begin{tikzpicture}[node distance=0.5cm]
\tikzset{
 mynd/.style={rectangle,rounded corners,draw=black, top color=white,
   bottom color=blue!50,very thick, inner sep=1em,
   minimum size=3em, text centered},
 myarr/.style={->, >=latex', shorten >=1pt, thick}
}
\node[mynd,top color=blue!30,bottom color=blue!30] (stable) {ТАБЛИЦА СИМВОЛОВ};
\node[above=2.5cm of stable] (dummy){};
\node[mynd,left=of dummy] (lexer) {ЛЕКСЕР};
\node[mynd,right=of dummy] (parser) {ПАРСЕР};
\node[right=of parser,text width=1.5cm] (fin) {Семантический анализ};
\node[left=of lexer,text width=1.5cm] (sta) {Исходный текст};

\draw[myarr,<->] (lexer) -- (stable);
\draw[myarr,<->] (parser) -- (stable);

\draw[myarr,transform canvas={yshift=0.3cm}] (lexer.east) -- (parser.west) node[midway,above]{токен};
\draw[myarr,<-,transform canvas={yshift=-0.3cm}] (lexer.east) -- (parser.west) node[midway,below]{lex};
\draw[myarr] (sta.center) -- (lexer.west);
\draw[myarr] (parser.east) -- (fin.center);
\end{tikzpicture}

%\includegraphics[scale=0.5]{lexer_role.png}
\end{frame}

\begin{frame}
  \frametitle{Токен, лексама и шаблон}
  \begin{description}
  \item[Токен] -- пара: имя и, опционально, прикреплённое значение. Имя -- это элемент некоторого конечного множества.
  \item[Лексема] -- последовательность знаков, соответствующая определённому токену.
  \item[Шаблон] -- описание всех лексем, соответствующих токенам с одним и тем же именем.
  \end{description}
\end{frame}

\begin{frame}
  \frametitle{Примеры}
  \begin{table}
    \begin{tabular}{l|p{4.5cm}|p{4cm}}
      \hline
      {\bf Имя токена} & {\bf Шаблон} & {\bf Лексемы} \\ \hline \hline
      BOOLEAN & ``\#t'' или ``\#f'' & \#t, \#f \\ 
      LPAREN & ``('' или ``['' & (, [ \\ 
      SYMBOL & набор букв и цифр, содержащий букву & alala, 3k34dk, if, let \\ 
      NUMBER & непустой набор цифр и точки & 32, 0.0, .121, 3.14 \\
      STRING & всё, кроме "\, окружённое " & "мечты" \, "error"\\
      SPACE & непустая последовательность пробелов & \\
    \end{tabular}
  \end{table}
\end{frame}

\begin{frame}
  \frametitle{Атрибуты токенов}
  \begin{block}{}
    (format \#t ``$\sim$a'' 18.3)
  \end{block}
  \begin{itemize}
  \item ( LPAREN )
  \item ( SYMBOL, индекс ``format'' в таблице символов )
  \item ( SPACE ) 
  \item ( BOOLEAN, true ) 
  \item ( SPACE ) 
  \item ( STRING, индекс ``$\sim$a'' в таблице строк ) 
  \item ( SPACE ) 
  \item ( NUMBER, 18.3 ) 
  \item ( RPAREN ) 
  \end{itemize}
\end{frame}

\begin{frame}
  \frametitle{Лексические ошибки}
  \begin{itemize}
  \item Некоторые ошибки лексер не способен обнаружить, например:
    \begin{itemize}
    \item (let ((a 3))) a) $\rightarrow$ лишняя скобка
    \item (/ 1 ``2'') $\rightarrow$ неправильный тип
    \end{itemize}
  \item Однако некоторые, может:
    \begin{itemize}
    \item "abcd  $\langle$ EOF $\rangle$ $\rightarrow$ незаконченная строка
    \item 13a $\rightarrow$ недопустимая конструкция в Си
    \end{itemize}
  \item Такие ошибки как правило сигнализируются когда ни один шаблон не может отыскать лексему в тексте
  \end{itemize}
\end{frame}

\begin{frame}
  \frametitle{Восстановление после ошибок}
  \begin{description}
  \item[Паника] Игнорирование всех последующих знаков. До конца файла, либо до некоторого знака (например пробела)
  \item[Догадка]
    \begin{itemize}
    \item Удаление одного знака из входного потока,
    \item Вставка надостающего знака во входной поток
    \item Замена одно знака другим
    \item Перестановка двух соседних знаков
    \end{itemize}
    и повторная попытка найти лексему.
  \end{description}
\end{frame}

\begin{frame}
  \frametitle{Общий механизм анализа}
  \begin{itemize}
    \item Лексер считывает и запоминает по одному знаку.
    \item Когда информации становится достаточно, он передаёт парсеру очередной токен.
    \item Иногда лексеру требуется знать последующие символы, чтобы определить текущий токен. Например:
      \begin{itemize}
      \item В Fortran: DO 5 I = 1.25 $\leftrightarrow$ DO 5 I = 1,25
      \item В Си: -, <, =
      \end{itemize}
      Для этого применяется двухбуферная схема.
  \end{itemize}
\end{frame}

\begin{frame}
  \frametitle{Определения}
  \begin{description}
  \item[Алфавит] -- конечное множество знаков. Обозначим $\Sigma$.
  \item[Язык] -- набор строк знаков из алфавита $\Sigma$.
  \item[Грамматика] -- набор правил, выделяющий из всех последовательностей алфавитных знаков,
    строки принадлежащие языку.
  \item[Функция значания L(e) = M] -- язык M соответствующий грамматике e.
  \end{description}
  \begin{block}{}
    Грамматика задаёт только один язык. Но язык может быть определён несколькими грамматиками.
    L(e) - инъекция.
  \end{block}
\end{frame}

\begin{frame}
  \frametitle{Регулярные грамматики}
  \begin{itemize}
  \item Задают языки, включающие:
    \begin{itemize}
      \item Пустую строку $\varepsilon$: \{``''\} или $\varnothing$
      \item Одиночные символы из алфавита $\Sigma$ : \{``a''\}, \{``b''\}
      \item Объединение: $A + B = \{a | a\in A\} \cup \{b | b\in B\}$
      \item Конкатенация: $AB = \{ab | a\in A \wedge b\in B\}$
      \item Повторение: $A$*$ = \bigcup\limits_{i\geq0}A^i, \{A^0=\varepsilon, A^i={A\dotso A}\ i$раз$\}$
    \end{itemize}
  \item Контекстно-свободные
    \begin{itemize}
    \item C++ шаблон: A$\textless$B$\textless$C$\textgreater \textgreater$ a;
    \item C++ сдвиг:  C$\textgreater \textgreater$ a;
    \end{itemize}
  \item Задаются регулярными выражениями
  \item Задаются кончеными автоматами
  \end{itemize}
\end{frame}

\begin{frame}
  \frametitle{Регулярные выражения}
  \begin{itemize}
  \item База:
    \begin{itemize}
      \item Пустая строка $\varepsilon$: $L(\varepsilon) = \{\varnothing\}$
      \item Один символ $a\in \Sigma$: $L(\#\backslash a) = \{"a"\}$
    \end{itemize}
  \item Композиция:
    \begin{itemize}
      \item Объединение: $L(A + B) = L(A) \cup L(B)$
      \item Конкатенация: $L(AB) = \{ab|a\in L(a) \wedge b\in B \}$
      \item Повторение: $L(A^*) = \bigcup\limits_{i\geq0}A^i$
    \end{itemize}
    \hrule
  \item Расширения:
    \begin{itemize}
      \item $A^+=AA^*$
      \item $A?=A+\varepsilon$
      \item $A^k=\underbrace{AA\dotsm A}_{k\ \text{раз}}$
    \end{itemize}
  \end{itemize}
\end{frame}

\begin{frame}
  \frametitle{POSIX нотация}
  \begin{itemize}
    \item $```` \equiv \varepsilon$
    \item $``abc'' \equiv \{``a''\}|\{``b''\}|\{``c''\}$
    \item $(A) \equiv A$
    \item $A|B \equiv A+B$
    \item $AB$
    \item $. \equiv \Sigma \backslash\{$``перевод строки''$\}$
    \item $[abc] \equiv ``a``|``b``|``c``$
    \item $[j-p] \equiv\{$все буквы коды которых лежат между j и p включительно $\}$
    \item $[^\wedge...] \equiv \Sigma$ кроме [...]
    \item $A\{n\} \equiv A^n$
    \item $A\{n,m\} \equiv A^n + A^{n+1} + \dotsb + A^m$
    \item $A\{,m\} \equiv A\{0,m\} ; A\{n,\} \equiv A^nA^*$
  \end{itemize}
\end{frame}

\begin{frame}
  \frametitle{Примеры}
  \begin{itemize}
    \item Цифра: '0'|'1'|'2'|'3'|'4'|'5'|'6'|'7'|'8'|'9' $\equiv [0-9]$
    \item Целое неотрицательное число($Z^+_0$): \only<2->{$[0-9]^{+}$}
    \item Целое число($Z$): \only<3->{$'-'?[0-9]^+$}
    \item Десятичное дробное число: \only<4->{$'-'?[0-9]^{*}{'.'}[0-9]^+$}
    \item Идентификатор в Си: \only<5->{$[a-zA-Z\_][0-9a-zA-Z\_]^*$}
  \end{itemize}
\end{frame}

\begin{frame}
  \begin{center}
    \structure{{\Huge А имплементация?}\\
      \vspace{1cm}
      {\Large Для строки s и выражения e, $s\in L(e)$?}}
  \end{center}
\end{frame}

\begin{frame}
  \frametitle{Конечный автомат}
    \begin{itemize}
    \item Входной алфавит $\Sigma$
    \item Конечный набор состояний $S$\\
       \tikz \node[state] {$q_i$};
    \item Начальное состояние $n\in S$\\
      \tikz \node[state,initial,initial text=] {$n$};
    \item Набор конечных состояний $F \subseteq S$\\
      \tikz \node[state,accepting] {$q_k$};
    \item Набор переходов из состояния $s_1\in S$ в состояние $s_2\in S$ по входу $a\in \Sigma$:
      $s_1 \xrightarrow{a} s_2$\\
      \begin{tikzpicture}
        \node[state] (s1) {$s_1$};
        \node[state] (s2) [node distance=2cm, right of=s1] {$s_2$};
        \draw[->,thick,bend left] (s1) edge node [above] {a} (s2);
      \end{tikzpicture}
    \end{itemize}
\end{frame}

\begin{frame}
  \frametitle{Таблица и диаграмма переходов}
  \begin{tikzpicture}[-latex,node distance=2cm,auto,initial text=]
    \node[state,initial]  (q0)                     {$q_0$};
    \node[state]          (q1) [above right of=q0] {$q_1$};
    \node[state]          (q2) [below right of=q0] {$q_2$};
    \node[state]          (q3) [right of=q1]       {$q_3$};
    \node[state]          (q4) [right of=q2]       {$q_4$};
    \node[state,accepting](q5) [below right of=q3] {$q_5$};

    \path (q0) edge              node {b} (q1)
               edge              node {b} (q2)
          (q1) edge              node {c} (q3)
               edge              node {d} (q4)
          (q2) edge              node {a} (q4)
               edge [loop below] node {b} (q2)
          (q3) edge              node {c} (q5)
               edge [loop above] node {a} (q3)
          (q4) edge              node {d} (q5);
  \end{tikzpicture}
  \begin{tabular}{l|| c | c | c | c }
        & a             & b             &    c         &   d\\
    \hline
    $q_0$ & $\varnothing$ & $q_1$, $q_2$  & $\varnothing$ & $\varnothing$\\
    $q_1$ & $\varnothing$ & $\varnothing$ & $q_3$         & $q_4$\\
    $q_2$ & $q_4$         & $q_2$         & $\varnothing$ & $\varnothing$\\
    $q_3$ & $q_3$         & $\varnothing$ & $q_5$         & $\varnothing$\\
    $q_4$ & $\varnothing$ & $\varnothing$ & $\varnothing$ & $q_5$\\
    $q_5$ & $\varnothing$ & $\varnothing$ & $\varnothing$ & $\varnothing$
  \end{tabular}
\end{frame}

\begin{frame}
  \frametitle{множественные и $\varepsilon$- переходы}
  \begin{tikzpicture}[-latex,node distance=3cm, auto]
    \node[state] (q0)                    {$q_0$};
    \node[state] (q1)[right of=q0]       {$q_1$};
    \node[state] (q2)[right of=q1]       {$q_2$};
    \node[state] (q3)[above right of=q2] {$q_3$};
    \node[state] (q4)[below right of=q2] {$q_4$};
    \node[state] (q5)[right of=q2]       {$q_5$};
    \path (q0) edge node {$\varepsilon$} (q1)
          (q2) edge node {a}             (q3)
               edge node {a}             (q4)
               edge node {a}             (q5);
  \end{tikzpicture}
\end{frame}

\begin{frame}
  \frametitle{ДКА и НКА}
  \begin{block}{Не детерменированный конечный автомат (НКА)}
    Конечный автомат, который может одновременно быть в нескольких состояниях.\\
    Может содержать:
    \begin{itemize}
      \item $\varepsilon$-переходы
      \item множественные переходы
    \end{itemize}
  \end{block}
  НКА как правило лаконичнее.
  \begin{block}{Детерменированный конечный автомат (ДКА)}
    Конечный автомат, находящийся в каждый момент времени только в одном состоянии.\\
    {\bf Не} содержит:
    \begin{itemize}
      \item $\varepsilon$-переходы
      \item множественные переходы
    \end{itemize}
  \end{block}
  НКА может быть преобразован в ДКА принимающий тотже язык.
\end{frame}

\begin{frame}
  \frametitle{Общая схема генерации сканера}
  \begin{center}
    \begin{tikzpicture}[start chain=going below]
      \node[on chain,join] {Лексическая спецификация};
      \node[on chain,join] {Набор регулярных выражений};
      \node[on chain,join] {НКА};
      \node[on chain,join] {ДКА};
      \node[on chain,join] {Табличная имплементация ДКА};
    \end{tikzpicture}
  \end{center}
\end{frame}

\begin{frame}
  \frametitle{Построение НКА для регулярного выражения}
  Автоматы для базовых елементов:
  \begin{itemize}
    \item $\varepsilon$ (пустая строка):\\
      \begin{tikzpicture}[-latex,node distance=2cm,auto,initial text=]
        \node[state,initial]  (q0)                     {$q_0$};
        \node[state,accepting](q1) [right of=q0]       {$q_1$};
        \path (q0) edge              node {$\varepsilon$} (q1);
      \end{tikzpicture}
    \item a$\in \Sigma$:\\
      \begin{tikzpicture}[-latex,node distance=2cm,auto,initial text=]
        \node[state,initial]  (q0)                     {$q_0$};
        \node[state,accepting](q1) [right of=q0]       {$q_1$};
        \path (q0) edge              node {a} (q1);
      \end{tikzpicture}
  \end{itemize}
  \begin{block}{Общее обозначение НКА для выражения E}
    \begin{center}
      \begin{tikzpicture}[-latex]
        \draw ellipse (2cm and 0.8cm) node {E} ++(1cm,0) node[state,accepting]{};
      \end{tikzpicture}
    \end{center}
  \end{block}
\end{frame}

\begin{frame}
  \frametitle{Построение НКА для регулярного выражения}
  Комбинации:
  \begin{itemize}
    \item $AB$:\\
      \begin{tikzpicture}[-latex,initial text=,auto]
        \draw (3cm,0) ellipse (2cm and 0.8cm) node {A} ++(1cm,0) node[state] (q0){}
               ++(4cm,0) ellipse (2cm and 0.8cm) node {B} ++(1cm,0) node[state,accepting]{};
        \draw (0,0) edge (1cm,0)
              (q0) edge node {$\varepsilon$} (6cm,0);
      \end{tikzpicture}
    \item $A + B$:\\
    \begin{tikzpicture}[-latex,initial text=,auto]
      \draw (3cm,-1cm) ellipse (2cm and 0.8cm) node {A} ++(1cm,0) node[state] (qa){}
            ++(-1cm,2cm) ellipse (2cm and 0.8cm) node {B} ++(1cm,0) node[state] (qb){};
      
      \node [state,initial] (q0) at (0,0) {};
      \node [state,accepting] (q1) at (6cm,0){};
      \draw (q0) edge node {$\varepsilon$} (1cm,-1cm)
                 edge node {$\varepsilon$} (1cm,1cm)
            (qa) edge node {$\varepsilon$} (q1)
            (qb) edge node {$\varepsilon$} (q1);
    \end{tikzpicture}
  \end{itemize}
\end{frame}

\begin{frame}
  \frametitle{Построение НКА для регулярного выражения}
  \begin{itemize}
    \item $A^*$:\\
      \begin{tikzpicture}[-latex,initial text=,auto]
        \draw (4cm,0) ellipse (2cm and 0.8cm) node {A} ++(1cm,0) node[state] (qa){};
        \node [state,initial] (q0) at (0,0) {};
        \node [state,accepting] (q1) at (8cm,0){};
      \draw (q0) edge node {$\varepsilon$} (2cm,0)
                 edge [bend left] node {$\varepsilon$} (q1)
            (qa) edge node {$\varepsilon$} (q1)
                 edge [bend left=45] node {$\varepsilon$} (q0);
      \end{tikzpicture}
  \end{itemize}
\end{frame}

\begin{frame}
  \frametitle{Пример}
  Выражение: {\LARGE $(a+b^*)c$}\\
  (Для удобства, вершины помечены)\\
  \begin{tikzpicture}[-latex,initial text=,auto,node distance=1.5cm]
    \node[state,initial] (q0) {$q_0$};
    \node[state,above right of=q0] (q1) {$q_1$};
    \node[state,right of=q1] (q2) {$q_2$};
    \node[state,below right of=q0] (q3) {$q_3$};
    \node[state,right of=q3] (q4) {$q_4$};
    \node[state,right of=q4] (q5) {$q_5$};
    \node[state,right of=q5] (q6) {$q_6$};
    \node[state,above right of=q6] (q7) {$q_7$};
    \node[state,right of=q7] (q8) {$q_8$};
    \node[state,accepting,right of=q8] (q9) {$q_9$};
    \draw (q0) edge node {$\varepsilon$} (q1)
               edge node {$\varepsilon$} (q3)
          (q1) edge node {a}             (q2)
          (q2) edge node {$\varepsilon$} (q7)
          (q3) edge node {$\varepsilon$} (q4)
               edge [bend left] node {$\varepsilon$} (q6)
          (q4) edge node {b}             (q5)
          (q5) edge node {$\varepsilon$} (q6)
               edge [bend left] node {$\varepsilon$} (q3)
          (q6) edge node {$\varepsilon$} (q7)
          (q7) edge node {$\varepsilon$} (q8)
          (q8) edge node {c}             (q9);
  \end{tikzpicture}
\end{frame}

\begin{frame}
  \frametitle{Преобразование НКА в ДКА}
  \begin{block}{$\varepsilon$-замыкание}
    {\bf $\varepsilon$-замыкание} --- Множество состояний НКА, в которые можно перейти из данного по цепочке $\varepsilon$-переходов.
    \hrule
    $\varepsilon\-closure(q_i)=\{q_i\}\cup\{\varepsilon\-closure(q_j) | \exists q_i\xrightarrow{\varepsilon}q_j\}$
  \end{block}
  Пусть S -- множество состояний НКА, s -- его начальное состояние, F -- все конечные.
  \begin{itemize}
    \item Каждое состояние ДКА (X) --- это подмножество (S) состояний НКА. $X\subseteq S$
    \item Начальное состояние: $\varepsilon closure(s)$
    \item Конечные состояния: $\{X | X\cap F \neq \varnothing\} $ 
    \item Переходы: $\exists X\xrightarrow{a} Y \Leftrightarrow
      Y = \bigcup\limits_{q_i} \varepsilon closure(q_i), q_i\in X$
  \end{itemize}
\end{frame}

\begin{frame}
  \frametitle{Пример преобразования НКА в ДКА}
  \only<1>{
  \begin{tikzpicture}[-latex,initial text=,auto,node distance=1.5cm]
    \node[state,initial] (q0) {$q_0$};
    \node[state,above right of=q0] (q1) {$q_1$};
    \node[state,right of=q1] (q2) {$q_2$};
    \node[state,below right of=q0] (q3) {$q_3$};
    \node[state,right of=q3] (q4) {$q_4$};
    \node[state,right of=q4] (q5) {$q_5$};
    \node[state,right of=q5] (q6) {$q_6$};
    \node[state,above right of=q6] (q7) {$q_7$};
    \node[state,right of=q7] (q8) {$q_8$};
    \node[state,accepting,right of=q8] (q9) {$q_9$};
    \draw (q0) edge node {$\varepsilon$} (q1)
               edge node {$\varepsilon$} (q3)
          (q1) edge node {a}             (q2)
          (q2) edge node {$\varepsilon$} (q7)
          (q3) edge node {$\varepsilon$} (q4)
               edge [bend left] node {$\varepsilon$} (q6)
          (q4) edge node {b}             (q5)
          (q5) edge node {$\varepsilon$} (q6)
               edge [bend left] node {$\varepsilon$} (q3)
          (q6) edge node {$\varepsilon$} (q7)
          (q7) edge node {$\varepsilon$} (q8)
          (q8) edge node {c}             (q9);
  \end{tikzpicture}
  }
  \only<2>{
    \begin{tikzpicture}[-latex,initial text=,auto,node distance=1.5cm]
      \node[state,fill=red!40,initial] (q0) {$q_0$};
      \node[state,fill=red!40,above right of=q0] (q1) {$q_1$};
      \node[state,right of=q1] (q2) {$q_2$};
      \node[state,fill=red!40,below right of=q0] (q3) {$q_3$};
      \node[state,fill=red!40,right of=q3] (q4) {$q_4$};
      \node[state,right of=q4] (q5) {$q_5$};
      \node[state,fill=red!40,right of=q5] (q6) {$q_6$};
      \node[state,fill=red!40,above right of=q6] (q7) {$q_7$};
      \node[state,fill=red!40,right of=q7] (q8) {$q_8$};
      \node[state,fill=black!20,accepting,right of=q8] (q9) {$q_9$};
      \draw (q0) edge node {$\varepsilon$} (q1)
                 edge node {$\varepsilon$} (q3)
            (q1) edge node {a}             (q2)
            (q2) edge node {$\varepsilon$} (q7)
            (q3) edge node {$\varepsilon$} (q4)
                 edge [bend left] node {$\varepsilon$} (q6)
            (q4) edge node {b}             (q5)
            (q5) edge node {$\varepsilon$} (q6)
                 edge [bend left] node {$\varepsilon$} (q3)
            (q6) edge node {$\varepsilon$} (q7)
            (q7) edge node {$\varepsilon$} (q8)
            (q8) edge node {c}             (q9);
    \end{tikzpicture}
  }
  \only<3>{
    \begin{tikzpicture}[-latex,initial text=,auto,node distance=1.5cm]
      \node[state,fill=red!40,initial] (q0) {$q_0$};
      \node[state,fill=red!40,above right of=q0] (q1) {$q_1$};
      \node[state,green!40!black,right of=q1] (q2) {$Q_2$};
      \node[state,fill=red!40,below right of=q0] (q3) {$q_3$};
      \node[state,fill=red!40,right of=q3] (q4) {$q_4$};
      \node[state,right of=q4] (q5) {$q_5$};
      \node[state,fill=red!40,right of=q5] (q6) {$q_6$};
      \node[state,green!40!black,fill=red!40,above right of=q6] (q7) {$Q_7$};
      \node[state,green!40!black,fill=red!40,right of=q7] (q8) {$Q_8$};
      \node[state,fill=black!20,accepting,right of=q8] (q9) {$q_9$};
      \draw (q0) edge node {$\varepsilon$} (q1)
                 edge node {$\varepsilon$} (q3)
            (q1) edge node {a}             (q2)
            (q2) edge node {$\varepsilon$} (q7)
            (q3) edge node {$\varepsilon$} (q4)
                 edge [bend left] node {$\varepsilon$} (q6)
            (q4) edge node {b}             (q5)
            (q5) edge node {$\varepsilon$} (q6)
                 edge [bend left] node {$\varepsilon$} (q3)
            (q6) edge node {$\varepsilon$} (q7)
            (q7) edge node {$\varepsilon$} (q8)
            (q8) edge node {c}             (q9);
    \end{tikzpicture}
  }
  \only<4->{
    \begin{tikzpicture}[-latex,initial text=,auto,node distance=1.5cm]
      \node[state,fill=red!40,initial] (q0) {$q_0$};
      \node[state,fill=red!40,above right of=q0] (q1) {$q_1$};
      \node[state,green!40!black,right of=q1] (q2) {$Q_2$};
      \node[state,fill=red!50!blue!60,below right of=q0] (q3) {$q_3$};
      \node[state,fill=red!50!blue!60,right of=q3] (q4) {$q_4$};
      \node[state,fill=blue!40,right of=q4] (q5) {$q_5$};
      \node[state,fill=red!50!blue!60,right of=q5] (q6) {$q_6$};
      \node[state,green!40!black,fill=red!50!blue!60,above right of=q6] (q7) {$Q_7$};
      \node[state,green!40!black,fill=red!50!blue!60,right of=q7] (q8) {$Q_8$};
      \node[state,fill=black!20,accepting,right of=q8] (q9) {$q_9$};
      \draw (q0) edge node {$\varepsilon$} (q1)
                 edge node {$\varepsilon$} (q3)
            (q1) edge node {a}             (q2)
            (q2) edge node {$\varepsilon$} (q7)
            (q3) edge node {$\varepsilon$} (q4)
                 edge [bend left] node {$\varepsilon$} (q6)
            (q4) edge node {b}             (q5)
            (q5) edge node {$\varepsilon$} (q6)
                 edge [bend left] node {$\varepsilon$} (q3)
            (q6) edge node {$\varepsilon$} (q7)
            (q7) edge node {$\varepsilon$} (q8)
            (q8) edge node {c}             (q9);
    \end{tikzpicture}
  }
  \onslide<5>{
  \makebox[\textwidth][r]{
    \begin{tikzpicture}[-latex,initial text=,auto,node distance=2.5cm]
      \node[state,fill=red!40,initial]    (q1)
                             {$q_{0,1,3,4,6,7,8}$};
      \node[state,fill=blue!40]           (q2)
         [below right of=q1] {$q_{3,4,5,6,7,8}$};
      \node[state,green!40!black]          (q3)
         [above right of=q1] {$Q_{2,7,8}$};
      \node[state,accepting,fill=black!20](q4)
         [above right of=q2] {$q_9$};
      \draw (q1) edge              node {b} (q2)
                 edge              node {a} (q3)
                 edge              node {c} (q4)
            (q2) edge              node {c} (q4)
                 edge [loop right] node {b} (q2)
            (q3) edge              node {c} (q4);
    \end{tikzpicture}
  }}
\end{frame}

%\fi

\begin{frame}
  \frametitle{Табличная имплементация}
  \begin{columns}
    \column{0.55\textwidth}
    \begin{table}
      \begin{tabular}{l|| c | c | c}
        A     & a             & b            &    c \\
        \hline
        $q_1$ & $q_3$         & $q_2$         & $q_4$ \\
        $q_2$ & $\varnothing$ & $q_2$         & $q_4$ \\
        $q_3$ & $\varnothing$ & $\varnothing$ & $q_4$  \\
        $q_4$ & $\varnothing$ & $\varnothing$ & $\varnothing$\\
      \end{tabular}
    \end{table}
    (define (go state input)\\
    \ \ (let ((next (gethash '(state input))))\\
    \ \ \ \ \ \ (if next\\
    \ \ \ \ \ \ \ \ \ (go next (move (forward input)))\\
    \ \ \ \ \ \ \ \ \#f)))\\
    (define (accept input)\\
    \ \ (let ((rez (go n input)))\\
    \ \ \ \ (and rez (accepting? rez))))\\

    \column{0.45\textwidth}
    \begin{tikzpicture}[-latex,initial text=,auto,node distance=2.5cm]
      \node[state,fill=red!40,initial]    (q1) {$q_1$};
      \node[state,fill=blue!40]           (q2) [below right of=q1] {$q_2$};
      \node[state,green!40!black]         (q3) [above right of=q1] {$q_3$};
      \node[state,accepting,fill=black!20](q4) [above right of=q2] {$q_4$};
      \draw (q1) edge              node {b} (q2)
                 edge              node {a} (q3)
                 edge              node {c} (q4)
            (q2) edge              node {c} (q4)
                 edge [loop below] node {b} (q2)
            (q3) edge              node {c} (q4);
    \end{tikzpicture}
  \end{columns}
\end{frame}
%Когда дойдёт до лексера, заметить, что он матчит самую длинную лексему, и из одинаковых выбирает по приоритету
% + обнаружение ошибок - выражение допускающее всё, с наименьшим приоритетом

\end{document}


