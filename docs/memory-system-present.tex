\documentclass[16pt,pdf,unicode]{beamer}

\usepackage[utf8]{inputenc}
\usepackage[english,russian]{babel}
\usepackage{listings}
\usepackage{tikz}

\usetikzlibrary{positioning}
\usetikzlibrary{calc} % for manimulation of coordinates
\tikzset{
    invisible/.style={opacity=0},
    visible on/.style={alt=#1{}{invisible}},
    alt/.code args={<#1>#2#3}{%
      \alt<#1>{\pgfkeysalso{#2}}{\pgfkeysalso{#3}} % \pgfkeysalso doesn't change the path
    },
  }
  
% Scheme listings
\lstdefinelanguage{Scheme}{
  morekeywords=[1]{define, define-syntax, define-macro, lambda, define-stream, stream-lambda},
  morekeywords=[2]{begin, call-with-current-continuation, call/cc,
    call-with-input-file, call-with-output-file, case, cond,
    do, else, for-each, if,
    let*, let, let-syntax, letrec, letrec-syntax,
    let-values, let*-values,
    and, or, not, delay, force,
    quasiquote, quote, unquote, unquote-splicing,
    map, fold, syntax, syntax-rules, eval, environment, query },
  morekeywords=[3]{import, export},
  alsodigit=!\$\%&*+-./:<=>?@^_~,
  sensitive=true,
  morecomment=[l]{;},
  morecomment=[s]{\#|}{|\#},
  morestring=[b]",
  basicstyle=\small\ttfamily,
  keywordstyle=\bf\ttfamily\color[rgb]{0,.3,.7},
  commentstyle=\color[rgb]{0.133,0.545,0.133},
  stringstyle={\color[rgb]{0.75,0.49,0.07}},
  upquote=true,
  breaklines=true,
  breakatwhitespace=true,
  literate=*{`}{{`}}{1}
}

% Escape to latex from lstlisting
\lstset{basicstyle=\small\ttfamily,
numbers=left,
escapeinside=||
}

\usetheme{Boadilla}
\setbeamercovered{transparent}
%\lstset{ language=Lisp }

\title[Организация памяти]{Теория и практика компилляции программ: среда времени исполнения}
\author[Д. Анисимов]{Денис Анисимов}
\institute[МФТИ]{Московский физико-технический университет\\
	{\tiny государственный университет}\\}
\date[16.02.2013]{МФТИ, 16 февраля 2013 года}

\begin{document}

\begin{frame}[plain]
\setcounter{framenumber}{0}
\titlepage
\end{frame}

\section{Введение}
\begin{frame}
\frametitle{Поддержка времени исполнения}
\begin{itemize}
  \item Для реализации статических конструкций языка во время исполнения необходимы дополнительные действия, определяемые логикой машинного исполнения.
  \item Связь имени в исходном языке с данными в машине представляет собой выделение и освобождение памяти.
  \item Поддержка работы с памятью осуществляется средой времени исполнения
  \begin{itemize}
    \item Состоит из программ подгружаемых с исполняемым кодом.
    \item Определяется семантикой процедур.  
  \end{itemize}
\end{itemize}
\end{frame}

\section{Процедуры}
\begin{frame}[fragile]
\frametitle{Процедуры}
\begin{lstlisting}[language=Scheme,numbers=left, stepnumber=1, numbersep=1pt]
(define add
  (lambda (num1 num2)
    (+ num1 num2)))

	(add 1 2)
\end{lstlisting}
\begin{itemize}
  \item \textit{Определение процедуры}(1-3): связь идентификатора с инструкцией(тело процедуры)
  \item \textit{Вызов процедуры}(5): исполнение тела процедуры. Каждое выполнение процедуры - \textit{активация}
  \item \textit{Формальные параметры}(num1 num2): идентификаторы специального характера
  \item \textit{Фактические параметры}(1 2): аргументы, передаваемые процедуре и замещающие формальные параметры.
\end{itemize}
\end{frame}

\begin{frame}
\frametitle{Деревья активации}
Время жизни активации процедур имеет вложенный характер: при вызове процедуры \textit{a} из \textit{b} процедура \textit{b} закончится раньше чем \textit{a}.
Наглядное представление процесса создания активаций - дерево.
\begin{center}
\begin{figure}
\begin{tikzpicture}[every node/.style=level 1/.style={sibling distance=30mm},level 2/.style={sibling distance=30mm}]
\node{q(1,9)}
  child { 
    node[left=1.7cm] {q(1,3)} 
      child { node{q(1,0)}}
      child {
        node {q(2,3)}
          child { node{q(2,1)} }
          child { node{q(3,3)} }
      }
  }
  child { 
    node[right=1.7cm] {q(5,9)}
      child { node {q(5,5)} }
      child {
        node {q(7,9)}
          child { node{q(7,7)} }
          child { node{q(9,9)} }
      }
  }
;
\end{tikzpicture}
\end{figure}
\end{center}
\end{frame}

\begin{frame}
\frametitle{Стек управления}
В процессе исполнения программы дерево активаций проходится обходом в глубину. Текущие рабочие активации составляют стек.
\begin{center}
\begin{figure}
\begin{tikzpicture}[every node/.style=level 1/.style={sibling distance=30mm},level 2/.style={sibling distance=30mm}]
\tikzstyle{na}=[opacity=0.3,dashed]
\node{q(1,9)}
  child { 
    node[left=1.7cm] {q(1,3)} 
      child[na] { node[na]{q(1,0)}}
      child {
        node {q(2,3)}
          child { node{q(2,1)} }
          child[na] { node[na]{q(3,3)} }
      }
  }
  child[na] { 
    node[na,right=1.7cm] {q(5,9)}
      child[na] { node[na] {q(5,5)} }
      child[na] {
        node[na] {q(7,9)}
          child[na] { node[na]{q(7,7)} }
          child[na] { node[na]{q(9,9)} }
      }
  }
;
\end{tikzpicture}
\end{figure}
\end{center}
\end{frame}

\begin{frame}
\frametitle{Продолжение}
\begin{itemize}
  \item Продолжение  - представление состояния исполнения программы на данный момент.
  \item Поддержка обращений позволяет обобщённо реализовать некоторые шаблоны программирования - исключения, сопрограммы и др.
  \item Организация памяти для языков непосредственно поддерживающий продолжения отличается от традиционных.
\end{itemize}
\end{frame}

\begin{frame}
\frametitle{Спагетти стек}
Для реализации продолжений необходимо уметь хранить состояния стека управления. Один из способов - реализация стека в виде динамического односвязного списка.
\begin{center}
\begin{figure}
\begin{tikzpicture}[every node/.style=level 1/.style={sibling distance=30mm},level 2/.style={sibling distance=30mm}]
\tikzstyle{na}=[opacity=0.3,dashed]
\node{q(1,9)}
  child { 
    node[left=1.7cm] {q(1,3)} 
      child[na] { node[na]{q(1,0)}}
      child {
        node {q(2,3)}
          child { node[draw](n1){q(2,1)} }
          child { node[draw](n2){q(3,3)} }
      }
  }
  child { 
    node[right=1.7cm] {q(5,9)}
      child[na] { node[na] {q(5,5)} }
      child {
        node {q(7,9)}
          child { node(n3){q(7,7)} }
          child[na] { node[na]{q(9,9)} }
      }
  }
;
\node at ($(n1)+(0.0,-0.6)$){saved};
\node at ($(n2)+(0.0,-0.6)$){saved};
\node at ($(n3)+(0.0,-0.6)$){current};
\end{tikzpicture}
\end{figure}
\end{center}
\end{frame}

\section{Организация памяти}
\begin{frame}
\frametitle{Классификация памяти}
\begin{itemize}
	\item Память используется для:
	\begin{itemize}
	  \item<1-2> Целевого кода
	  \item<1,3> Объектов данных
	  \item<1,4> Стека управления
	\end{itemize}
\end{itemize}
\begin{center}
	\begin{tabular}{|c|}
	\hline
	\uncover<1-3>{Код} \\ \hline	
	\uncover<1,3>{Статические данные} \\	\hline
	\uncover<1,3,4>{Стек} \\	\hline
	\uncover<1,3,4>{$\downarrow$}	\\
	\uncover<1,3,4>{$\uparrow$} \\ \hline
	\uncover<1,3,4>{\alert<4>{Куча}}\\ \hline
  \end{tabular}
\end{center}
\end{frame}

\begin{frame}
\frametitle{Записи активаций}
Запись активации или кадр - непрерывный блок памяти содержащий информацию необходимую для однократного исполнения процедуры.
\begin{center}
  \begin{tabular}{|c|}
  \hline
  \uncover<1,8>{Возвращаемое значение} \\ \hline  
  \uncover<1,7>{Фактические параметры} \\ \hline
  \uncover<1,6>{Связь управления} \\ \hline
  \uncover<1,5>{Связь доступа} \\ \hline
  \uncover<1,4>{Состояние машины} \\ \hline
  \uncover<1,3>{Локальные данные} \\ \hline
  \uncover<1,2>{Временные значения} \\ \hline
  \end{tabular}
\end{center}
\begin{center}
\only<2>{Значения появляющиеся в процессе вычислений}
\only<3>{Локальные данные для данной процедуры}
\only<4>{Текущий контекст исполнения, например регистры, счётчик команд(зависит от конкретной машины)}
\only<5>{Ссылка для доступа к нелокальным переменным}
\only<6>{Ссылка на активацию вызывающей процедуры}
\only<7>{Фактические параметры передающиеся процедуре}
\end{center}
\end{frame}

\begin{frame}
\frametitle{Размещение локальных данных}
\begin{itemize}
  \item Память под статически известные данные размечается компилятором в процессе обработки объявлений.
  \item Память под них выделяется динамически, но относительное положение известно статически.
  \item В процессе размещения учитываются ограничения целевой машины и соображения производительности.
\end{itemize}
\end{frame}

\section{Стратегии выделения памяти}
\begin{frame}
\frametitle{Статическое распределение}
\begin{itemize}
  \item Компилятор статически распределяет адреса всех объектов в памяти
  \item Недостатки
  \begin{itemize}
    \item Необходимо знать размеры объектов и ограничения памяти в процессе компиляции.
    \item Ограничение рекурсии
    \item Нет динамических структур данных
  \end{itemize}
\end{itemize}
\end{frame}

\begin{frame}
\frametitle{Стековое распределение}
\begin{itemize}
  \item Данные динамически располагаются на стеке в процессе вычислений и вызова процедур.
  \item Локальные переменные разных активаций располагаются в отдельных областях памяти и уничтожаются при выходе из процедуры.
  \item Используется для представления стека управления.
\end{itemize}
\end{frame}

\begin{frame}
\frametitle{Распределение памяти в куче}
\begin{itemize}
  \item В Scheme значения локальных переменных сохраняются по окончанию активации, а также возможно возобновление исполнения из сохранённого состояния стека. Поэтому стековое распределение памяти для хранения активаций не подходит.
  \item Распределение в куче позволяет динамически выделять память и освобождать её в \textit{произвольном} порядке.
  \item Недостатки:
  \begin{itemize}
    \item Эффективное использование кучи требует сложного пакета подержки времени исполнения.
    \item Необходимо освобождать неиспользуемую память, вручную или автоматически.
  \end{itemize}
\end{itemize}
\end{frame}

\section{Доступ к нелокальным именам}
\begin{frame}
\frametitle{Лексическая область видимости и вложенные процедуры}
\begin{itemize}
  \item Каждая блочная область видимости в компиляторе представляется таблицей символов и связывается с областью памяти для локальных данных.
  \item Поддержка вложенности процедур и принципа ближайшего определения осуществляется ссылкой на область локальных данных охватывающей процедуры(лексического блока)
  \item Доступ к нелокальным данным в процессе исполнения происходит по переходу по ссылкам + смещение в области локальных данных.
\end{itemize}
\end{frame}

\section{Передача параметров}
\begin{frame}
\frametitle{Передача параметров}
\begin{itemize}
  \item Семантически передача параметров может происходить по значению, по ссылке или обоими способами.
  \item Фактический механизм передачи параметров определяется целевой машиной и соображениями оптимизации.
\end{itemize}
\end{frame}

\end{document}