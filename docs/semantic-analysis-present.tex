\documentclass[16pt,pdf,unicode]{beamer}

\usepackage[utf8]{inputenc}
\usepackage[english,russian]{babel}
\usepackage{listings}
\usepackage{tikz}
\usepackage{color}
\usepackage{geometry}
\usepackage{amsmath}
\usepackage{multicol}

\usetikzlibrary{positioning}
\usetikzlibrary{automata}
\usetikzlibrary{chains}
\usetikzlibrary{shapes,arrows}
\usetikzlibrary{calc} % for manimulation of coordinates

\usetheme{Boadilla}
\setbeamercovered{transparent}

\lstset{ language=Lisp,
         showspaces=false,
         morekeywords={define, begin, car, cdr}}

\title[Семантический анализ]{Теория и практика компилляции программ: семантический анализ}
\author[А.Ю. Заостровных]{Арсений Юрьевич Заостровных}
\institute[МФТИ]{Московский физико-технический институт\\
	{\tiny государственный университет}\\}
\date[2013]{МФТИ, 9 февраля 2013 года}

\begin{document}

%\iffalse

\begin{frame}[plain]
\maketitle
\end{frame}

\begin{frame}
  \frametitle{Положение семантического анализа}
\begin{tikzpicture}[node distance=0.4cm]
\tikzset{
 mynd/.style={rectangle,rounded corners,draw=black, top color=white,
   bottom color=blue!30,very thick, inner sep=0.2cm,
   minimum size=1cm, text centered},
 myarr/.style={->, >=latex', shorten >=1pt, very thick}
}
\node[mynd] (lexer) {ЛЕКСЕР};
\node[mynd,right=of lexer] (parser) {ПАРСЕР};
\node[mynd,right=of parser,bottom color=blue!50,inner sep=0.1cm,text width=3.5cm] (seman) {СЕМАНТИЧЕСКИЙ  АНАЛИЗАТОР};
\node[right=of seman,text width=1.3cm,text centered,inner sep=0cm] (fin) {Генерация кода};
\node[left=of lexer,text width=1.3cm,text centered,inner sep=0cm] (sta) {Исходный текст};

\draw[myarr] (sta.center) -- (lexer.west);
\draw[myarr] (lexer) -- (parser);
\draw[myarr] (parser) -- (seman);
\draw[myarr] (seman.east) -- (fin.center);
\end{tikzpicture}
\end{frame}

\begin{frame}
  \frametitle{Задачи семантического анализа}
  \begin{itemize}
  \item Уяснение значения контекстно зависимых конструкций
    \begin{itemize}
    \item Вывод типов сложных выражений
    \item Разрешение вызовов перегруженных функций
    \item Выполнение необходимых преобразований типов
    \item Построение графа наследования
    \item Построение областей видимости идентификаторов
    \end{itemize}
  \item Нахождение ''смысловых`` ошибок, таких, как:
    \begin{itemize}
    \item Неответствие типов
    \item Неоднозначное определение классов
    \item Нарушение прав доступа
    \item Изменение констант
    \item Инстанцирование абстрактных классов
    \item Вызов функции с неправильными аргументами
    \end{itemize}
  \item Набор задач полностью определяется языком
  \end{itemize}
\end{frame}

\begin{frame}
  \frametitle{Общий алгоритм}
\begin{itemize}
  \item Обход AST в глубину
  \item Вывод свойств кода локально на каждом узле дерева на основании:
    \begin{itemize}
      \item информации переданной от предка
      \item результатов анализа потомков
    \end{itemize}
  \item По типу узла применяется соответствующее правило вывода
\end{itemize}
\end{frame}

\begin{frame}
  \frametitle{Область видимости}
  \begin{itemize}
    \item {\bf Привязка символа (binding)} --- ассоциация символа (имени) с конкретным объектом программы (кодом или данными)
    \item {\bf Область видимости (scope)} --- часть кода в которой символ определён и имеет одну и ту же привязку (binding)
    \begin{itemize}
      \item Глобальная
      \item Модульная
      \item Коллективная (shared)
      \item Локальная
      \begin{itemize}
        \item {\bf Лексическая} --- имя ассоциируется с текстуально ближайшим определением 
        \item {\bf Динамическая} --- имя ассоциируется с последним определению (вверх по стеку вызовов)
      \end{itemize}
    \end{itemize}
  \end{itemize}
\end{frame}

\begin{frame}[fragile]
  \frametitle{Лексическая и динамическая области видимости}
\begin{lstlisting}
  (define a 1)
  (define (foo)
    (write a) (set! a 2))
  (define (bar)
    (let ((a 3)) (foo)))
    
  (bar)
  (write a)
\end{lstlisting}
Вывод программы зависит от обласити видимости принятой в языке:
\begin{itemize}
  \item Лексическая: 1 2
  \item Динамическая: 3 1
\end{itemize}
\end{frame}

\begin{frame}
  \frametitle{Поддержка областей видимости}
\begin{block}{}
  {\bfЛексическая} область видимости требует анализа на стадии компиляции. Для разрешения неоднозначностей применяются ``таблицы символов''.
\end{block}
\begin{exampleblock}{}
  {\bfДинамическая} - требует только поддержки во время исполнения. Используется поиск вверх по стеку вызовов. Для этого нужны специальным образом организованные активации функций, либо дополнительные структуры данных.
\end{exampleblock}
\end{frame}

\begin{frame}[fragile]
  \frametitle{Таблица символов}
\begin{columns}
\column{0.5\linewidth}
\begin{lstlisting}
  (lambda (a b c)
    (+ a
       (let (a)
         (let (d e)
           (let (c)
             (+ a b c))))))
\end{lstlisting}
\column{0.5\linewidth}
\begin{tikzpicture}[node distance=0.1cm]
  \tikzset{
    nd/.style={rectangle,draw,text width=4cm,text centered}
  }
  \node[nd] (frst) {$a\rightarrow1\ \ b\rightarrow2\ \ c\rightarrow3$};
  \node[nd,below=of frst] (scnd) {$a\rightarrow4$};
  \node[nd,below=of scnd] (thrd) {$d\rightarrow5\ \ e\rightarrow6$};
  \node[nd,below=of thrd] (frth) {$c\rightarrow7$};
\end{tikzpicture}
\end{columns}
\end{frame}

\begin{frame}
  \frametitle{Типы}
\begin{exampleblock}{Тип это}
  \begin{itemize}
    \item Множество значений
    \item Множество операций, определённых на этих значениях
  \end{itemize}
\end{exampleblock}
\begin{block}{}
  Типы существуют только в языке высокого уровня.
\end{block}
\begin{block}{Задачи}
  \begin{itemize}
    \item разрешение неоднозначных выражений
    \item проверка корректности программы
    \item дополнительная информация для оптимизатора
  \end{itemize}
\end{block}
\end{frame}

\begin{frame}
  \frametitle{Варианты типизации}
\begin{description}
\item[Статическая]: все проверки и выводы производятся один раз, на этапе компиляции (C, Java, Haskell)
  \begin{itemize}
    \item быстрее исполнение
    \item более защищенно
  \end{itemize}
\item[Динамическая]: проверка и разрешение неоднозначностей происходят во время исполнения (Perl, Scheme, Shell)
  \begin{itemize}
    \item быстрее прототипирование
    \item более гибко
  \end{itemize}
\end{description}
\end{frame}

\begin{frame}
  \frametitle{Правила вывода}
\begin{itemize}
  \item Общая форма: ``Если верна гипотеза, то верно и заключение''
  \item Строчная нотация: $H_1\land H_2\land ...\land H_n \Rightarrow C$, здесь:
    \begin{itemize}
      \item $a\land b$ -- конъюнкция
      \item $a\Rightarrow b$ -- импликация (если a, то b)
      \item $H_i$ и $C$ -- гипотезы и заключение. Могут иметь форму $e_i:T_i$
      \item $e:T$ -- выражение е принадлежит типу Т.
    \end{itemize}
  \item Столбцовая нотация:\[\frac{\vdash H_1 ... \vdash H_n}{\vdash C}\], где:
    \begin{itemize}
      \item $\vdash$ означает доказуемо
      \item конъюнкции подразумеваются
    \end{itemize}
\end{itemize}
\end{frame}

\begin{frame}
  \frametitle{Примеры}
\begin{block}{Исходные правила}
  \[\frac{\text{а - целое число}}{\vdash a:\mathbb{Z}}\]
  \[\frac{\vdash e_1:\mathbb{Z} \vdash e_2:\mathbb{Z}}{\vdash e_1*e_2:\mathbb{Z}}\]
\end{block}
\begin{block}{Вывод}
  Выражение $1 * 2 * 3$
  \huge{
  \[\frac{\frac{\text{1 - целое}}{\vdash 1:\mathbb{Z}}\ \frac{\frac{\text{2-целое}}{\vdash 2:\mathbb{Z}}\ \frac{\text{3 - целое}}{\vdash 3:\mathbb{Z}}}{\vdash 2 * 3 :\mathbb{Z}}}{\vdash 1 * 2 * 3 :\mathbb{Z}}\]}
\end{block}
\end{frame}

\begin{frame}
  \frametitle{Окружение типов}
\begin{itemize}
  \item {\bfСвободная переменная} -- переменная, не определяемая в рассматриваемом выражении:\\
    (let ((a b)) (+ a b)) -- здесь b - свободна везде, а а - свободна только во внутреннем выражении.
  \item {\bfОкружение типов} О -- функция ассоциирующая свободные переменные с типами. (Таблица символов)
  \item $O \vdash e:T $ означает: В предположении, что типы свободных переменных даны через О, е доказуемо имеет тип Т.
  \item $O[T/x]$ означает окружение О, дополненное ассоциацией переменной x с типом Т:
    \begin{itemize}
      \item $O[T/x](x) = T$
      \item $y\neq x \Rightarrow O[T/x](y)=O(y)$
    \end{itemize}
  \item Новое правило:
\[\frac{O(x)=T}{O\vdash x:T}\]
\end{itemize}
\end{frame}

%\fi

\begin{frame}
  \frametitle{Примеры}
  \[\frac{O\vdash e_1:\mathbb{Z}\ \ O\vdash e_2:\mathbb{Z}}{O\vdash e_1*e_2:\mathbb{Z}}\]
  \\
  \\
  \[\frac{\begin{aligned} &O\vdash e_1:T_1 \\ &O[T_1/x]\vdash e_2:T_2
    \end{aligned}}{O\vdash (let\ ((x\ e_1))\ e_2):T_2}\]
\end{frame}

\begin{frame}
  \frametitle{Уточнение типов($\leq$)}
\begin{itemize}
  \item $X\leq X$
  \item $X\leq Y$ если X наследует (или уточняет) Y
  \item $X\leq Y \land Y\leq Z \Rightarrow X\leq Z$
  \item Точная верхняя грань ТВГ(X,Y) -- наиболее точный тип Z: $X \leq Z \land Y \leq Z$. Как правило это младший общий предок в древе типов (древе наследования).
  \item примеры:
\end{itemize}
\begin{multicols}{2}
  \[\frac{\begin{aligned} &O\vdash e_0:Bool\\
      &O\vdash e_1:T_1\\
      &O\vdash e_2:T_2
  \end{aligned}}{O\vdash e_0 ? e_1 : e_2 :\text{ТВГ}(T_1,T_2)}\]
  \[\frac{\begin{aligned} &O\vdash e:T_1 \\ &O(x)=T_2 \\ &T_1 \leq T_2
  \end{aligned}}{O\vdash x=e:T_2}\]
\end{multicols}
\end{frame}

\begin{frame}
  \frametitle{Диспетчеризация методов}
\begin{itemize}
  \item Специальный тип для функции или метода: $(T_1,...,T_n\rightarrow T_0)$
  \item Подчас используется отдельное пространство имен
  \item Обозначение: $M(C,f)=(T_1,...,T_n\rightarrow T_0)$ означает, что в классе C есть метод f c сигнатурой: $f(x_1:T_1,...,x_n:T_n): T_0$. М -- окружение методов.
\end{itemize}
\[\frac{\begin{aligned}
    O,M&\vdash e_x:T_x\\
    O,M&\vdash e_1:T_1\\
    &\hdots\\
    O,M&\vdash e_n:T_n\\
    M(T_x,f)=(&T_1^{'},\hdots,T_n^{'}\rightarrow T_0)\\
    T_i\leq &T_i^{'}, 1\leq \leq n
  \end{aligned}}{O,M\vdash e_x.f(e_1,...,e_n):T_0}\]
\end{frame}

\begin{frame}[fragile]
  \frametitle{Пример реализации}
\begin{exampleblock}{}
\[\frac{\begin{aligned} &O,M\vdash e_1:T_1 \\ &O[T_1/x],M\vdash e_2:T_2
\end{aligned}}{O,M\vdash (let\ ((x\ e_1))\ e_2):T_2}\]
\end{exampleblock}
\begin{lstlisting}
(define (type-check O M let-expr)
  (let ((T (type-check O M
                       (initializer-part let-expr)))
        (x (var-part let-expr)))
    (type-check (cons `(,x ,T) O) M
                (body-part let-expr))))
\end{lstlisting}
\end{frame}

\begin{frame}
  \frametitle{Итоги}
\begin{itemize}
  \item Статическое соответствие типов проверяется на этапе компиляции
  \item Проверка делается обходом в глубину и применением локальных правил
  \item Правила вывода недостающих типов -- компактные утверждения позиционной логики
  \item Типы таких конструкций как переменные и вызовы моделируются с помощью окружения
\end{itemize}
\end{frame}

\end{document}


