\documentclass[16pt,pdf,unicode]{beamer}

\usepackage[utf8]{inputenc}
\usepackage[english,russian]{babel}
\usepackage{listings}
\usepackage{tikz}
\usepackage{color}
\usepackage{geometry}


\usetikzlibrary{chains}
\usetikzlibrary{shapes,arrows}

\usetheme{Boadilla}
\setbeamercovered{transparent}
\lstset{ language=Lisp }

\title[Знакомство со Scheme]{Знакомство с языком scheme: основные особенности}
\author[А.Ю. Заостровных]{Заостровных Арсений Юрьевич}
\institute[МФТИ]{Московский физико-технический университет\\
	{\tiny государственный университет}\\}
\date[2012]{МФТИ, 15 сентября 2012 года}

\begin{document}

%\iffalse 

\begin{frame}[plain]
\maketitle
\end{frame}

\begin{frame}
\frametitle{Почему scheme?}
\begin{itemize}
\item Простота языка: не требуется тонны синтаксических правил
\item Богатство современными технологиями позволит детально разобраться в них.
\item Язык является основой знаменитого курса MIT - SICP (Structure and Interpretation of Computer Programs)
\end{itemize}
\end{frame}

\begin{frame}
\frametitle{Scheme - мощный язык}
\begin{itemize}
\item Макросы
\item Автоматическое управление памятью
\item Функции высших порядков
\item Ленивые вычесления
\item ... и многое другое
\vspace{15 mm}
\item Квайн: ((lambda (x) `(,x ',x)) '(lambda (x) `(,x ',x)))
\item<2> Квайн для придирчивых:\\
  ((lambda (x) (write `(,x ',x))) '(lambda (x) (write `(,x ',x))))
\end{itemize}
\end{frame}

\begin{frame}
\frametitle{Scheme - простой язык}
\begin{itemize}
\item Парсер занимает меньше 200 строк кода
\item Всего 2 грамматических элемента:
  \begin{enumerate}
    \item атом
    \item список
  \end{enumerate}
\item Никаких выведений типов: динамическая типизация
\item Беззаботное программирование: автоматическая сборка мусора
\end{itemize}
\end{frame}

\section{Синтаксис}

\begin{frame}
\frametitle{Основные лексемы}
\begin{itemize}
\item Различные скобки: (,),[,],\#(, \#vu8(,
\item Символы: sym, add, +, \%\#?!, ...
  \begin{itemize}
    \item Регистроиндифферентны
  \end{itemize}
\item Числа: 42, 3.1415, -.6L0, 1.3 - 2.7i, -1.2@15
\item Булевы значения: \#f, \#t
\item Знаки \#$\backslash$0, \#$\backslash\backslash$, \#$\backslash$newline, \#$\backslash$\ , \#$\backslash$03bb
\item Строки ''string'',  '''', ''$\backslash$x41;bc''
\item Кавычки ', `, ,, ,@
\end{itemize}
\end{frame}

\begin{frame}
  \frametitle{Грамматика}
  \begin{itemize}
    \item Атом a, +, \#f, 84, ...
    \item Точечная пара(cons-ячейка) (a . b)
    \item Список (), (a b), (a . (b . ())), (a b c 33 \#f ()), ((()()))
    \item Вектор \#(1 3 () \#t)
    \item Битовый вектор \#vu8(0 1 1 0 1)
  \end{itemize}
\end{frame}

\begin{frame}
  \frametitle{Семантика}
  Общее правило: голова списка - исполнитель, тело - его аргументы.\\
  Три вида исполнителей:
  \begin{itemize}
   \item Специальные формы
     \begin{itemize}
       \item (define a 3) $\rightarrow$ привязка значения 3 к a
       \item (if a b c) $\rightarrow$ выполнение только одной ветви
     \end{itemize}
   \item Функции
     \begin{itemize}
     \item (cons a b) $\rightarrow$ (a . b)
     \item (+ 1 2 3 4) $\rightarrow$ 10
     \end{itemize}
   \item Макросы
     \begin{itemize}
     \item (+= a b) $\rightarrow$ (set! a (+ a b))
     \item Перевычисляются
     \end{itemize}
  \end{itemize}
\end{frame}

\begin{frame}
\frametitle{Соглашение о наименовании}
\begin{itemize}
  \item Все символы (идентификаторы) записываются в нижнем регистре, слова разделяются знаком ``-''
  \item Предикаты оканчиваются на вопросительный знак - ``?''
  \item Процедура изменяющая свои аргументы - на ``!''
\end{itemize}
\end{frame}

\section{Исполнение}
\begin{frame}
\frametitle{Основные понятия}
\begin{itemize}
\item Ссылка
\item Область видимости
\item Время жизни
\item Вложенный блок
\item Предметно ориентированный язык
\item Контекст
\end{itemize}
\end{frame}

\begin{frame}
\frametitle{Специальные формы}
\begin{itemize}
  \item {\bf define} добавляет переменную в текущий блок
  \item {\bf if} альтернирует последовательность исполнения
  \item {\bf set!} меняет значение существующей переменной
  \item {\bf lambda} создаёт функцию
  \item {\bf quote, etc.} Блокирует вычисление некоторых выражений
  \item {\bf define-syntax} создаёт трансформатор выражений (макрос)
  \item {\bf begin} склеивает несколько выражений в одно
\end{itemize}
\end{frame}

\begin{frame}
\frametitle{Функции}
\begin{block}{Встроенные функции}
cons, car, cdr, +, =, eq?, vector-ref, make-string, port?, 
write, read, apply, map, string=?, string-set! ...
\end{block}
\begin{block}{Функции высших порядков}
(map sq '(1 2 3 4)) $\rightarrow$ (1 4 9 16)\\
(reduce + '(1 2 3 4)) $\rightarrow$ 10
\end{block}
\begin{block}{Определение собственной функции}
(define (sq x) (* x x))\\
(define foo (lambda (a b c) (= (+ (sq a) (sq b)) (sq c))))
\end{block}
\end{frame}

\begin{frame}
\frametitle{Рекурсия}
\begin{minipage}{0.5\linewidth}
  
  {\bf Реку́рсия} — процесс повторения элементов самоподобным образом.
  \begin{itemize}
  \item {\bf Хвостовая рекурсия} — специальный случай рекурсии, при котором рекурсивный вызов функцией самой себя является её последней операцией.
    
  \end{itemize}
\end{minipage}
\hspace{0.05\linewidth}
\begin{minipage}{0.38\linewidth}
          \includegraphics[scale=0.3]{recursion.jpg}
\end{minipage}
\end{frame}
\begin{frame}
\frametitle{Хвостовая рекурсия}
\begin{columns}
  \column{0.45\textwidth}
\only<1>{
  \begin{block}?

(define (factorial n)\\
\ \ (define (fac-times n acc)\\
\ \ \ \ (if (= n 0)\\
\ \ \ \ \ \ acc\\
\ \ \ \ \ \ (fac-times (- n 1) (* acc n))))\\
\ \ (fac-times n 1))

  \end{block}
}
\only<2>{
  \begin{exampleblock}{Хвостовая рекурсия}

(define (factorial n)\\
\ \ (define (fac-times n acc)\\
\ \ \ \ (if (= n 0)\\
\ \ \ \ \ \ acc\\
\ \ \ \ \ \ (fac-times (- n 1) (* acc n))))\\
\ \ (fac-times n 1))
  \end{exampleblock}
}

  \column{0.45\textwidth}
\only<1>{
  \begin{block}?
(define (factorial n)\\
\ \ (if (= n 0)\\
\ \ \ \ 1\\
\ \ \ \ (* n (factorial (- n 1)))))\\
  \end{block}
}
\only<2>{
  \begin{alertblock}{{\emНе} хвостовая рекурсия}
(define (factorial n)\\
\ \ (if (= n 0)\\
\ \ \ \ 1\\
\ \ \ \ (* n (factorial (- n 1)))))\\
  \end{alertblock}
}
\end{columns}
\end{frame}

\begin{frame}
  \frametitle{Функциональный стиль}
  \begin{block}{Сумма чисел от 0 до 150}
    \begin{columns}
      \column{0.45\textwidth}
      (let ((sum 0))\\
      \ \ (do ((i 0 (+ i 1))\\
      \ \ \ \ \ \ ((< i 150) sum)\\
      \ \ \ \ (set! sum (+ sum i)))))\\
      \column{0.45\textwidth}
      (reduce + (range 0 150))
    \end{columns}
  \end{block}
  \begin{block}{Длина списка}
    \begin{columns}
      \column{0.45\textwidth}
      (define (length lst)\\
      \ \ (let ((len 0))\\
      \ \ \ \ \ (do ((c lst (cdr c)))\\
      \ \ \ \ \ \ \ \ \ ((null? c) len)\\
      \ \ \ \ \ \ \ (set! len (+ len 1))))) 
      \column{0.45\textwidth}
      (define (length lst)\\
      \ \ (if (null? lst)\\
      \ \ \ \ 0\\
      \ \ \ \ (+ 1 (length (cdr lst)))))
    \end{columns}
  \end{block}
\end{frame}

\begin{frame}
\frametitle{Множественные значения}
Множественные значения не допустимы в контексте, ожидающем единственное значение.
\begin{itemize}
  \item ({\bf values} a b c ...) -- Создание многозначной величины
  \item ({\bf let-values} (((a b c) (foo)) ...) ...) -- ``Потрошение'' множественного значения
  \item ({\bf call-with-values} generator receiver) -- Трансляция множественного значения, произведённого generator'ом, receiver'у.
\end{itemize}
\end{frame}

\begin{frame}
  \frametitle{Замыкание}
  {\bf \emph{Замыкание}} -- функция, в теле которой присутствуют ссылки на переменные, объявленные вне тела этой функции и не в качестве её параметров (а в окружающем коде).
\begin{exampleblock}{Пример}
  ({\bf define} (make-counter)\\
  \ \ ({\bf let} ((c 0))\\
  \ \ \ \ ({\bf values} ({\bf lambda} () ({\bf set!} c (+ c 1)))\\
  \ \ \ \ \ \ \ \ \ \ \ \ \ \ ({\bf lambda} () c))))\\
  \vspace{0.5cm}
  ({\bf let-values} (((incr1 val1) (make-counter))\\
  \ \ \ \ \ \ \ \ \ \ \ \ \ \ \ \ ((incr2 val2) (make-counter)))\\
  \ \ (incr1) (incr2) (incr1)\\
  \ \ (incr2) (incr2) ({\bf format} ``$\sim$ a, $\sim$ a'' (val1) (val2))) ; $\rightarrow$ 2, 3
\end{exampleblock}
\end{frame}

\begin{frame}
\frametitle{Ленивые вычисления (обещания)}
\begin{itemize}
\item ``Обещание'' - объект языка, содержащий всё необходимое для вычисления некоторого выражения.
\item ({\bf delay} выражение): создаёт ``обещание'' вычисления выражения.
\item ({\bf force} обещание): вычисляет ``обещание''. Если оно уже было однажды вычислено, использует кешированное значение.
\item {\bf promise?} -- предикат ``обещания''
\item {\bf promise-forced?} -- предикат кеша
\item {\bf promise-value} -- извлечение значения только из кеша.
\end{itemize}
Пример:\\
(force (delay (+ 1 2))) ;$\rightarrow$ 3
\end{frame}

%\fi

\section{Макросы}
\begin{frame}
  \begin{center}
    \structure{\Huge \insertsection}
  \end{center}
\end{frame}

\begin{frame}
\frametitle{Необходимость макросов}
\begin{itemize}
\item Изменение контекста\\
  (push! 'a lst) $\rightarrow$ (set! lst (cons 'a lst))
\item Последовательность исполнения\\
  (let ((a \#f) (b \#f))  (or (set! a \#t) (set! b \#t)) ) $\rightarrow$ \#t; a=\#t, b=\#f
\item Создание привязок\\
  (let ((x 0)) x) $\rightarrow$ ((lambda (x) x) 0)
\item Автоматическое управление ресурсами\\
  (with-file (file ``a.in'') (print (read file))) $\rightarrow$\\
  (let ((file (open-file ``a.in''))) (print (read file)) (close-file file))
\item Определение предметно-ориентированного языка\\
  (destructuring-bind (a (c . e))) (foo) (print c)) $\rightarrow$\\
  (let ((val (foo)))\\
  \ \ (let ((a (car val))\\
  \ \ \ \ \ \ \ \ (c (car (car (cdr val))))\\
  \ \ \ \ \ \ \ \ (e (cdr (car (cdr val)))))\\
  \ \ \ \ (print c)))
\end{itemize}
\end{frame}

%\fi

\begin{frame}
\frametitle{Раскрытие (подстановка) макроса}
\begin{block}{}
 (define-syntax f foo)
\end{block}
\vspace{0.5cm}

\tikzset{
	commonnode/.style={
		rectangle, 
		rounded corners, 
		draw=black, thick,
		minimum height=1.5cm, 
		text centered, 
		font=\smaller,
                fill=blue!10,
		on chain},
		every node/.style={text width=5.5cm},
		every join/.style={->, thick,shorten >=1pt},
}
\hyphenpenalty=10000 
\begin{tikzpicture}
\begin{scope}
[node distance=.5cm,	start chain=going right]
	\node[commonnode, join](first) {``Обертка'' аргументов (f~a~b) $\rightarrow$ \#<syntax~(a~b)>};
	\node[commonnode, join] (last){Вызов обработчика (foo~\#<syntax~(a~b)>) $\rightarrow$ \#<syntax s>} ;
\end{scope}%TODO: Запретить переносы в середине слов!!
\begin{scope}
[node distance=1.9cm,	start chain=going left]
	\node[commonnode, join, anchor=north west] (preob) at ([commonnode, yshift=-0.9cm] last.south west)
             {Подстановка результатов \#<syntax~s> на место вызова};
        \node[draw=black, thick, on chain, text badly centered,
              text width=2cm, fill=blue!10, inner sep=-0.1cm, diamond, join](ques)
             { Остались нераскрытые макросы? };
\end{scope}
\begin{scope}
	\node[commonnode, fill=green!10, anchor=north](finish) at ([commonnode, yshift=-0.2cm, xshift=1cm] ques.south)
             {Исполнение полученного кода};
\end{scope}
\begin{scope}
	\draw [->, thick] (last.south) -- (preob.north);
	\draw [->, thick] (ques) -- (finish);
	\draw [->, thick] (ques) -- (first);
\end{scope}
\end{tikzpicture}
\end{frame}

\begin{frame}
\frametitle{Пример макроса}
\begin{block}{}
  (define-syntax swap!\\
  \ \ (sc-macro-transformer\\
  \ \ \ \ (lambda (exp usage-env)\\
  \ \ \ \ \ \ (let ((a (cadr exp)) (b (caddr exp)))\\
  \ \ \ \ \ \ \ \ `(let ((tmp ,a)) (set! ,a ,b) (set! ,b tmp))))))
\end{block}
\begin{exampleblock}{Раскрытие}
\begin{itemize}
  \item (swap! x y) ; $\rightarrow$
  \item (let ((tmp x)) (set! x y) (set! y tmp))
\end{itemize}
\end{exampleblock}
\end{frame}

\begin{frame}
\frametitle{syntax-rules}
\begin{itemize}
  \item Гигиеничны:
    Автоматически распознают все символы не входящие в аргументы макросы, и переименовывает их.
  \item Прозрачны:
    Поддержывают лексическую область видимости.
  \item Лаконичны:
    Имеют собственный язык шаблонов.
  \item Закрыты:
    Нет возможности вызвать функцию, что позволяет разделить стадии компилляции и исполнения.
\end{itemize}
\end{frame}

\begin{frame}
\frametitle{Синтаксис syntax-rules}
\begin{block}{}
({\bf syntax-rules} (литералы ...)\\
\ \ ((образец замена) ...))
\end{block}
\begin{itemize}
  \item {\bf оригинал} -- выражение, подвергающееся раскрытию макроса
  \item {\bf литералы} -- символы в образце совпадающие точно по имени
  \item {\bf образец} -- дерево (список списков), состоящее из символов (идентификаторов, литералов и ``...'').
  \item {\bf замена} -- произвольное выражение, в котором идентификаторы из образца будут заменены на подвыражения из оригинала.
\end{itemize}
$\bullet$ В образце может встретиться символ ..., который трактуется особо - он собирает в себя остаток текущего списка.\\
$\bullet$ В замене, на место ... вклеивается соответствующий список из оригинала.
\end{frame}

\begin{frame}
\frametitle{Примеры}
\begin{block}{}
  (define-syntax swap!\\
  \ \ (syntax-rules ()\\
  \ \ \ \ ((swap! a b) (let ((tmp a)) (set! a b) (set! b tmp)))))
\end{block}

\begin{block}{}
  (define-syntax let\\
  \ \ (syntax-rules ()\\
  \ \ \ \ ((let ((var val)...) body ...)\\
  \ \ \ \ \ ((lambda (var ...) body ...) val ...))))
\end{block}

\end{frame}

\begin{frame}
\frametitle{Проблемы макроподстановок: захват}
\begin{block}{Определение}
(define-syntax swap!\\
\ \ (syntax-rules () ((swap! a b) (let ((tmp a)) (set! a b) (set! b tmp)))))
\end{block}
\begin{block}{}
  (swap! x tmp)
\end{block}
\centering{\huge{$\Downarrow$}}
\begin{columns}
  \column{0.45\textwidth}
\begin{alertblock}{Тривиальное раскрытие}
  (let ((tmp x))\\
  \ \ (set! x tmp)\\
  \ \ (set! tmp tmp))\\
  ; $\rightarrow$ значения x и tmp остались прежними.
\end{alertblock}
  \column{0.45\textwidth}
\begin{exampleblock}<2>{Гигиеничное раскрытие}
  (let ((tmp1 x))\\
  \ \ (set! x tmp)\\
  \ \ (set! tmp tmp1))
\end{exampleblock}
\end{columns}
\end{frame}

\begin{frame}
\frametitle{Проблемы макроподстановок: сокрытие}
\begin{block}{Определение}
  (define mem '())\\
  (define-syntax remember-if\\
  \ \ (syntax-rules () ((\_ cond val) (if cond (push! val mem)))))
\end{block}
\begin{block}{}
  (let ((mem \#(1 2))) (remember-if \#t mem))
\end{block}
\centering{\huge{$\Downarrow$}}
\begin{columns}
  \column{0.45\textwidth}
\begin{alertblock}{Тривиальное раскрытие}
  (let ((mem \#(1 2)))\\
  \ \ (if \#t (push! mem mem)))\\
  ; $\rightarrow$ Ошибка!
\end{alertblock}
  \column{0.45\textwidth}
\begin{exampleblock}<2>{Прозрачное раскрытие}
  (define mem1 mem)\\
  (let ((mem \#(1 2)))\\
  \ \ (if \#t (push! mem mem1)))
\end{exampleblock}
\end{columns}
\end{frame}

\begin{frame}
\frametitle{Остались неупомянутыми}
\begin{itemize}
\item Интернирование
\item Продолжения (continuations)
\item Исключительные ситуации
\item Ввод/вывод
\item Структуры
\item Что-то ещё?
\end{itemize}
\end{frame}

\end{document}


