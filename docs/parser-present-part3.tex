\documentclass[16pt,pdf,unicode]{beamer}

\usepackage[utf8]{inputenc}
\usepackage[english,russian]{babel}
\usepackage{listings}
\usepackage{algpseudocode}
\usepackage{tikz}
\usepackage{color}
\usepackage{geometry}
\usepackage{multirow}

\usetikzlibrary{positioning}
\usetikzlibrary{automata}
\usetikzlibrary{chains}
\usetikzlibrary{scopes}
\usetikzlibrary{arrows}
\usetikzlibrary{matrix}
\usetikzlibrary{fit}
\usetikzlibrary{backgrounds}
\usetikzlibrary{shapes.geometric}
\usetikzlibrary{shapes.multipart}
\usetikzlibrary{intersections}
\usetikzlibrary{calc} % for manimulation of coordinates
\tikzset{
    invisible/.style={opacity=0},
    visible on/.style={alt=#1{}{invisible}},
    alt/.code args={<#1>#2#3}{%
      \alt<#1>{\pgfkeysalso{#2}}{\pgfkeysalso{#3}} % \pgfkeysalso doesn't change the path
    },
  }

%
% Looooooong macro to draw nested math sets
%
\newcommand\drawnestedsets[4]{
  % initial position
  \def\position{#1}
  % number of sets
  \def\nbsets{#2}
  % list of sets
  \def\listofnestedsets{#3}
  % reversed list of colors
  \def\reversedlistofcolors{#4}

  % position and draw labels of sets
  \coordinate (circle-0) at (#1);
  \coordinate (set-0) at (#1);
  \foreach \set [count=\c] in \listofnestedsets {
    \pgfmathtruncatemacro{\cminusone}{\c - 1}
    % label of current set (below previous nested set)
    \node[below=3pt of circle-\cminusone,inner sep=0]
    (set-\c) {$\set$};
    % current set (fit current label and previous set)
    \node[circle,inner sep=0,fit=(circle-\cminusone)(set-\c)]
    (circle-\c) {};
  }

  % draw and fill sets in reverse order
  \begin{scope}[on background layer]
    \foreach \col[count=\c] in \reversedlistofcolors {
      \pgfmathtruncatemacro{\invc}{\nbsets-\c}
      \pgfmathtruncatemacro{\invcplusone}{\invc+1}
      \node[circle,draw,fill=\col,inner sep=0,
      fit=(circle-\invc)(set-\invcplusone)] {};
    }
  \end{scope}
}


\usetheme{Boadilla}
\setbeamercovered{transparent}
%\lstset{ language=Lisp }

\title[Синтаксический анализ]{Теория и практика компилляции программ: синтаксический анализ, часть III}
\author[Д. Анисимов]{Денис Анисимов}
\institute[МФТИ]{Московский физико-технический университет\\
	{\tiny государственный университет}\\}
\date[08.12.2012]{МФТИ, 08 декабря 2012 года}

\begin{document}

\begin{frame}[plain]
\setcounter{framenumber}{0}
\titlepage
\end{frame}

\section{Восходящий анализ}
\begin{frame}
\frametitle{Восходящий анализ}
\begin{itemize}
  \item Основной метод восходящего анализа - "перенос/свёртка"(shift-reduce)
  \item Идея:
    \begin{itemize}
      \item Будем строить дерево разбора начиная с листьев
      \item Каждый шаг \textit{свёртки} - замена некоторой подстроки, соответствующей правой части продукции, левой частью продукции. 
      \item В конце получим обращённое порождение входной строки.
    \end{itemize}
\end{itemize}
\end{frame}

\begin{frame}
\frametitle{Пример}
Грамматика:
\begin{align*}
  &S \rightarrow \alert<5>{aABe}\\
  &A \rightarrow \alert<3>{Abc} \mid \alert<2>{b}\\
  &B \rightarrow \alert<4>{d}
\end{align*}
Процесс свёртки:
\begin{align*}
\onslide
a{\alert<2>{b}}bcde\\
\pause
a\alert<3>{\alert<2>{A}bc}de\\
\pause
a\alert<3>{A}\alert<4>{d}e\\
\pause
\alert<5>{aA\alert<4>{B}e}\\
\pause
\alert<5>{S}\\
\end{align*}
Соответствующее порождение:
$$\onslide<5->S \rightarrow\onslide<4-> aABe \rightarrow\onslide<3-> aAde \rightarrow \onslide<2->aAbcde \rightarrow \onslide<1->abbcde $$
\end{frame}

\begin{frame}
\frametitle{Основы}
\begin{itemize}
  \item Основа - подстрока, совпадающая с правой частью продукции, свёртка которой в левую часть является одним шагом обращённого порождения.
  \item Для правосентенциальной формы $\gamma$, основа это:
  \begin{itemize}
    \item Продукция $A \rightarrow \beta$
    \item Позиция строки $\beta$ в $\gamma$
    \item При замене $\beta$ на $A$ получаем предыдущую правосентенциальную форму правого порождения $\gamma$
  \end{itemize}
  \item Пример:
    \begin{itemize}
      \item $S \xrightarrow[rm]{*} \alpha A \omega \xrightarrow[rm]{} \alpha \beta \omega$
      \item $A \rightarrow \beta$  в позиции $\alpha$ является основой строки $\alpha \beta \omega$
      \item $\omega$ содержит только терминалы
    \end{itemize}
\end{itemize}
\end{frame}

\begin{frame}
\frametitle{Обрезка основ}
\begin{itemize}
  \item Получение  обращённого правого порождения - обрезка основ.
  \item Если $\omega$ - предложение рассматриваемой грамматики, то $\omega = \gamma_n$, где $S = \gamma_0 \xrightarrow[rm]{} \gamma_1 \xrightarrow[rm]{} \dots \xrightarrow[rm]{} \gamma_n$
  \item Для получение $\gamma_{n-1}$ находим основу $\beta_n$ в $\gamma_n$ и заменяем её левой частью продукции $A_n \rightarrow \beta_n$ 
\end{itemize}
\end{frame}

\begin{frame}
\frametitle{Пример}
\begin{columns}
	\begin{column}{4cm}
		Грамматика:
		\begin{align*}
		&E \rightarrow E + E\\
		&E \rightarrow E * E\\
		&E \rightarrow (E)\\
		&E \rightarrow \textbf{id}
		\end{align*}
	\end{column}
	\begin{column}{6cm}
    \begin{tikzpicture}
		  \node {$E$}
		    [level distance=10mm]
		    child [visible on=<1-5>] {
		      node {$E$}
		        child [visible on=<1>] {node {$\textbf{id}_1$}}
		    }
		    child [visible on=<1-5>] {node {+}}
		    child [visible on=<1-5>] {
		      node {$E$}
		        child [visible on=<1-4>] {
		          node {$E$}
		            child [visible on=<1-2>] {node {$\textbf{id}_2$}}
		        }
		        child [visible on=<1-4>] {node {*}}
		        child [visible on=<1-4>] {
		          node {$E$}
		            child [visible on=<1-3>] {node {$\textbf{id}_2$}}
		        }
		    }
		  ;
		\end{tikzpicture}	
	\end{column}
\end{columns}
Входная строка:
$$\textbf{id}_1 + \textbf{id}_2 * \textbf{id}_3$$

\begin{tabular}{r|c|l}
\hline
Правосентенциальная форма & Основа & Сворачивающая продукция\\
\hline
\uncover<1->{$\textbf{id}_1 + \textbf{id}_2 * \textbf{id}_3$} & 
\uncover<1->{$\textbf{id}_1$} & 
\uncover<1->{$E \rightarrow \textbf{id}$} \\

\uncover<2->{$E + \textbf{id}_2 * \textbf{id}_3$} &
\uncover<2->{$\textbf{id}_2$} &
\uncover<2->{$E \rightarrow \textbf{id}$} \\

\uncover<3->{$E + E * \textbf{id}_3$}  &
\uncover<3->{$\textbf{id}_3$} & 
\uncover<3->{$E \rightarrow \textbf{id}$}\\

\uncover<4->{$E + E * E$} & 
\uncover<4->{$E * E$} & 
\uncover<4->{$E \rightarrow E * E$}\\

\uncover<5->{$E + E$} & 
\uncover<5->{$E + E$} & 
\uncover<5->{$E \rightarrow E + E$}\\

\uncover<6->{$E$} & & \\
\hline

\end{tabular}
\end{frame}

\begin{frame}
\frametitle{Стековая реализация ПС анализа}
\begin{itemize}
  \item Символы грамматики - стек, анализируемая строка - сходной буффер.
  \item Основные операции:
    \begin{description}
      \item[Перенос] - символ из входного буфера переносится на вершину стека.
      \item[Свёртка] - замена символов на вершине стека левой частью продукции.
      \item[Допуск/Ошибка]
    \end{description}
  \item Основное свойство такой реализации - основа всегда находится на вершине стека.
\end{itemize}
\end{frame}

\begin{frame}
\frametitle{Активные префиксы}
\begin{itemize}
  \item Префиксы правосентенциальных форм, являющиеся корректными состояниями стека в процессе работы ПС анализатора.
  \item Активные префиксы не выходят за правую границу крайней справа основы.
  \item Просканированная часть исходной строки не содержит ошибок только в том случае, если она может быть свёрнута в активный префикс.
\end{itemize}
\end{frame}

\section{LR-анализаторы}

\begin{frame}
\frametitle{LR-анализаторы}
\begin{description}
  \item[LR(k)-анализ] L - сканирование входной строки слева направо \\
                      R - построение обращённый правых порождений \\
                      k - число символов для принятия решения
\end{description}
\begin{itemize}
  \item Для большинства конструкций языков программирования возможно построить LR-грамматику
  \item Наиболее общий метод ПС-анализа без отката
  \item Достаточно трудоёмкий для ручного построения
  \item Существуют генераторы LR анализаторов
\end{itemize}
\end{frame}

\subsection{Алгоритм LR-анализа}
\begin{frame}
\frametitle{Схема LR-анализатора}
\begin{figure}
\tikzstyle{sym}=[node distance=-0.15mm,minimum size=0.6cm]
\tikzstyle{st}=[node distance=-0.15mm,minimum size=0.8cm]
\begin{tikzpicture}[start chain=1 going right, start chain=2 going below]
  \node[sym,draw,on chain=1] at (1.5cm,1.5cm) {$a_1$};
  \node[sym,draw,on chain=1] {$..$};
  \node[sym,draw,on chain=1] (s) {$a_i$};
  \node[sym,draw,on chain=1] {$..$};
  \node[sym,draw,on chain=1] {$a_n$};
  \node[sym,draw,on chain=1] {$\$$};
  \node[st,draw,on chain=2] at (0cm,0cm) (t) {$s_m$};
  \node[st,draw,on chain=2] {$X_m$};
  \node[st,draw,on chain=2] {$..$};
  \node[st,draw,on chain=2] {$\$$};
  \node[draw, text width=3cm, text badly centered] at (s |- t) (syn) {LR-анализатор};
  
  \node[rectangle split, rectangle split horizontal, 
        rectangle split parts=2, rectangle split part align={center},
        draw, anchor=north, minimum height=0.7cm,
        below=0.7cm of syn] (table)
  {
    \nodepart{one}
    action
    \nodepart{two} 
    goto
  };
  
  \node[right=0.7cm of syn] (exit) {Выход};
  \node[left of=t] {Стек};
  \node[left =1.2cm of s] {Вход};
  \draw[<-] (t.east) -- (syn.west);
  \draw[<-] (s.south) -- (syn.north);
  \draw[->] (syn.south) to [out=270,in=90] (table.two north);
  \draw[->] (syn.south) to [out=270,in=90] (table.one north);
  \draw[->] (syn.east) -- (exit.west);
  
\end{tikzpicture}
\end{figure}
\end{frame}

\begin{frame}
\frametitle{Общий алгоритм LR-анализа}
\textit{Конфигруция} LR-анализатора - совокупность стека и входного буфера:
    $$ s_0 X_1 s_1 X_2 \dots X_m s_m, \quad a_i a_{i+1} \dots a_n $$
Действия LR-анализатора
\begin{itemize}  
  \item $action[s_m,a_i] = "shift\:s"$
    $$ s_0 X_1 s_1 X_2 \dots X_m s_m a_i s, \quad a_{i+1} \dots a_n $$
  \item $action[s_m,a_i] = "reduce\:A \rightarrow \beta"$
    $$ s_0 X_1 s_1 X_2 \dots X_{m-r} s_{m-r} A s, \quad a_i a_{i+1} \dots a_n $$
    где
    \begin{itemize}
      \item $s = goto[s_{m-r},A]$
      \item $r$ - длина $\beta$
    \end{itemize}
  \item $action[s_m,a_i] = accept/reject$
\end{itemize}
\end{frame}

\subsection{SLR-анализаторы}

\begin{frame}
\frametitle{Типы LR-анализаторов}
\vspace*{-0.5cm}
\begin{center}
\begin{tikzpicture}
\drawnestedsets{0,0}{5}{SLR(k),LALR(k),LR(k),Unambiguous,All CFGs}{blue!20,cyan,red!40,magenta!20,red!80}
\end{tikzpicture}
\end{center}
\end{frame}

\begin{frame}
\frametitle{SLR-анализ}
\begin{itemize}
  \item Наиболее слабый(по числу грамматик)
  \item Наиболее простой для реализации
  \item Основная идея - на основе грамматики построить ДКА для распознавания активных префиксов
\end{itemize}
\end{frame}

\begin{frame}
\frametitle{LR(0)-пункт}
\begin{itemize}
  \item $LR(0)$-пункт грамматики $G$ - продукция $G$ с точкой в некоторой позиции в правой части
  \item Например, продукция $A \rightarrow XYZ$ даёт 4 пункта:
    \begin{itemize}
      \item $A \rightarrow \cdot XYZ$
      \item $A \rightarrow X \cdot YZ$
      \item $A \rightarrow XY \cdot Z$
      \item $A \rightarrow XYZ \cdot $
    \end{itemize}
  \item Продукция $A \rightarrow \varepsilon$ даёт только один пункт: $A \rightarrow \cdot$
\end{itemize}
\end{frame}

\begin{frame}
\frametitle{Каноническая система LR(0)-пунктов}
\begin{itemize}
  \item Расширенная грамматика $G'$:
    \begin{itemize}
      \item Дополнительный стартовый символ $S'$
      \item Продукция $S' \rightarrow S$
    \end{itemize}
  \item Функции
    \begin{itemize}
      \item Замыкания - $closure$
      \item Перехода - $goto$
    \end{itemize}
\end{itemize}
\end{frame}

\begin{frame}
\frametitle{Операция замыкания}
Если $I$ - некоторое множество пунктов грамматики $G$, то его замыкание $closure(I)$ строится по правилам:
\begin{itemize}
  \item Изначально $closure(I) = I$
  \item Если $\{A \rightarrow \alpha \cdot B \beta\} \in closure(I)$, и $\exists B\rightarrow \gamma$, то добавляем пункт $B \rightarrow \cdot \gamma$ в $closure(I)$
\end{itemize}
\pause
\begin{columns}
\begin{column}{0.5\textwidth}
\begin{align*}
  E' &\rightarrow E \\
  E &\rightarrow E + T \mid T \\
  T &\rightarrow T * F \mid F \\
  F &\rightarrow (E) \mid \textbf{id} \\
\end{align*}
$$ I = \{E' \rightarrow \cdot E \} $$
\end{column}
\begin{column}{0.5\textwidth}
\begin{align*}
closure(I) = \\
\{E' &\rightarrow \cdot E \\
  E &\rightarrow \cdot E + T \\
  E &\rightarrow \cdot T \\
  T &\rightarrow \cdot T * F \\
  T &\rightarrow \cdot F \\
  F &\rightarrow \cdot (E) \\
  F &\rightarrow \cdot \textbf{id}\}
\end{align*}
\end{column}
\end{columns}
\end{frame}

\begin{frame}
\frametitle{Операция перехода}
Если $I$ - множество пунктов, а $X$ - символ грамматики $G$, то $goto(I,X)$ определяется как замыкание всех пунктов $[A \rightarrow \alpha X \cdot \beta]$, таких, что $[A \rightarrow \alpha \cdot X \beta]$
\pause
\begin{columns}
\begin{column}{0.5\textwidth}
\begin{align*}
  E' &\rightarrow E \\
  E &\rightarrow E + T \mid T \\
  T &\rightarrow T * F \mid F \\
  F &\rightarrow (E) \mid \textbf{id} \\
\end{align*}
\begin{align*}
I = \{&[E' \rightarrow E \cdot], \\ &[E \rightarrow E \cdot + T] \}
\end{align*}
\end{column}
\begin{column}{0.5\textwidth}
\begin{align*}
goto(I,+) = \\
 \{E &\rightarrow E + \cdot T \\
  T &\rightarrow \cdot T * F \\
  T &\rightarrow \cdot F \\
  F &\rightarrow \cdot (E) \\
  F &\rightarrow \cdot \textbf{id}\}
\end{align*}
\end{column}
\end{columns}
\end{frame}

\begin{frame}
\frametitle{Построение множества пунктов}
\begin{itemize}
  \item Алгоритм построения:
  \begin{itemize}
    \item Начинаем со стартового пункта $C = closure(\{[S' \rightarrow \cdot S]\})$
    \item Для каждого элемента множества $C$ и каждого символа $X$ добавляем $goto(I,X)$ в $C$
  \end{itemize}
  \item Полученное множество множеств пунктов - состояния ДКА, функции $goto$ определяют переходы между состояниями
\end{itemize}
\end{frame}

\begin{frame}
\frametitle{Построение таблицы SLR-анализа}
\begin{itemize}
  \item Состояние ДКА == состояние анализатора
  \item Действия анализатора:
    \begin{itemize}
      \item Если $[A \rightarrow \alpha \cdot a \beta]\in I_i$ и $goto(I_i,a)=I_j$, то $action[i,a]$ - "перенос j"
      \item Если $[A \rightarrow \alpha \cdot]\in I_i$, то $action[i,a]$ - "свёртка $A \rightarrow \alpha$" для всех $a$ из $FOLLOW(A)$
      \item Если $[S' \rightarrow S \cdot ] \in I_I$, то $action[i,\$]$ - "допуск"
    \end{itemize}
  \item Таблица переходов $goto$ == таблица переходов ДКА
  \item Начальное состояние - множество пунктов, содержащее $[S' \rightarrow \cdot S]$
  \item Все неопределённые записи - "ошибка"
\end{itemize}
\end{frame}

\end{document}


