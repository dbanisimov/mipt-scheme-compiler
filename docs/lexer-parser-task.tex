\documentclass[twocolumn,a4,10pt]{article}

%AMS-TeX packages
\usepackage{amssymb,amsmath,amsthm} 
%geometry (sets margin) and other useful packages
\usepackage[margin=1in]{geometry}
\usepackage{graphicx,ctable,booktabs}
\usepackage[utf8]{inputenc}
\usepackage[english,russian]{babel}
\usepackage{listings}
\usepackage{tikz}
\usepackage{color}
\usepackage{geometry}
\usepackage[pdftex,
            pdfauthor={Arseniy Zaostrovnykh, Denis Anisimov},
            pdftitle={Lexer and Parser quiz},
            pdfsubject={MIPT-Scheme Compiler},
            pdfkeywords={mipt, scheme, compiler}]{hyperref}

\usetikzlibrary{positioning}
\usetikzlibrary{automata}

\lstset{ language=Lisp,
         showspaces=false,
         morekeywords={define, begin, car, cdr}}

%
%Redefining sections as problems
%
\makeatletter
\newenvironment{problem}{\@startsection
       {section}
       {1}
       {-.2em}
       {-3.5ex plus -1ex minus -.2ex}
       {2.3ex plus .2ex}
       {\pagebreak[3]%forces pagebreak when space is small; use \eject for better results
       \large\bf\noindent{Задача }
       }
       }
       {%\vspace{1ex}\begin{center} \rule{0.3\linewidth}{.3pt}\end{center}}
       \begin{center}\large\bf \ldots\ldots\ldots\end{center}}
\makeatother


%
%Fancy-header package to modify header/page numbering 
%
\usepackage{fancyhdr}
\pagestyle{fancy}
%\addtolength{\headwidth}{\marginparsep} %these change header-rule width
%\addtolength{\headwidth}{\marginparwidth}
\lhead{Задача \thesection}
\chead{} 
\rhead{\thepage} 
\lfoot{\small\scshape MIPTScheme} 
\cfoot{} 
\rfoot{\footnotesize iLab МФТИ} 
\renewcommand{\headrulewidth}{.3pt} 
\renewcommand{\footrulewidth}{.3pt}
\setlength\voffset{-0.25in}
\setlength\textheight{648pt}

%%%%%%%%%%%%%%%%%%%%%%%%%%%%%%%%%%%%%%%%%%%%%%%

%
%Contents of problem set
%    
\begin{document}

\title{Лексический анализ. Синтаксический анализ. \\
      {\small  MIPT Scheme compiler }}
%\author{Арсений Заостровных, Денис Анисимов}
\date{24.11.12}

\maketitle

\thispagestyle{empty}

\begin{problem}{\it Регулярные выражения}
Составте регулярные выражения для:
\begin{enumerate}
\item Шестнадцатиричного числа, начинающегося с \#x (\#x110, \#xA3f)
\item Времени в двенадцати часовом формате (12:00 AM, 09:32 PM, 01:33 AM)
\item Е-mail адреса (user.x@example.info)
\end{enumerate}
\end{problem}

\begin{problem}{\it Конечные автоматы}
Для конечного автомата:
  
      \begin{tikzpicture}[node distance=2cm,auto,initial text=]
        \node[state,initial]   (q1)                     {$q_1$};
        \node[state]           (q2) [right of=q1]       {$q_2$};
      	\node[state,accepting] (q3) [below right of=q2] {$q_3$};
      	\node[state,accepting] (q4) [left of=q3]        {$q_4$};
        \draw[->,thick,bend left] (q1) edge node {a} (q2);
        \draw[->,thick,bend left] (q2) edge node {b} (q1);
        \draw[->,thick,bend left] (q2) edge node {c} (q3);
        \draw[->,thick,bend left] (q3) edge node {d} (q4);
        \draw[->,thick,bend left] (q4) edge node {d} (q3);
        \draw[->,thick,bend left] (q4) edge node {d} (q1);
      \end{tikzpicture}

Напишите последовательность состояний и результат обработки следующих строк:
\begin{enumerate}
\item aaabacd
\item acddababac
\item abacdd
\end{enumerate}
$\bullet$ В чем различия детерменированного и недетерменированного конечных автоматов?
\end{problem}

\begin{problem}{\it Эквивалентность}
\begin{enumerate}
\item Нарисуйте конечный автомат, эквивалентный регулярному выражению $(a|(bc)^*)^+c$
\item Выпишите регулярное выражение, эквивалентное конечному автомату из задачи 2.
\end{enumerate}
\end{problem}

\begin{problem}{\it Неоднозначность}
  Рассмотрим грамматику:
    $$ S \rightarrow aSbS \: \mid \: bSaS \: \mid \: \varepsilon $$
    Покажите, что эта грамматика неоднозначна, построив два различных левых порождения для предложения $abab$.
\end{problem}

\begin{problem}{\it Левая рекурсия}
 Рассмотрим грамматику:
    \begin{align*}
      &E \rightarrow E + T \: \mid \: \ T \\
      &T \rightarrow \textbf{id} \: \mid \: (E) 
    \end{align*}
  Напишите грамматику корректно устраняющую левую рекурсию из данной.
\end{problem}

\begin{problem}{\it Метод рекурсивного спуска}
 Рассмотрим грамматику:
    \begin{align*}
      &E \rightarrow E' \mid E' + E \\
      &E' \rightarrow -E' \mid \textbf{id} \mid (E)
    \end{align*}
    И входную строку:
      $$ \textbf{id} + \textbf{id} $$
    \begin{enumerate}
      \item Измените исходную грамматику так, чтобы она была совместима с методом рекурсивного спуска.
      \item Напишите последовательность порождений, выполняемую в методе рекурсивного спуска с новой грамматикой. Укажите \textit{все} порождения, т.е. включая неверные, после которых был откат.    
    \end{enumerate}
   
\end{problem}

\end{document}
