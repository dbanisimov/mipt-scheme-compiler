\documentclass[twocolumn,10pt]{article}

%AMS-TeX packages
\usepackage{amssymb,amsmath,amsthm} 
%geometry (sets margin) and other useful packages
\usepackage[margin=1in]{geometry}
\usepackage{graphicx,ctable,booktabs}
\usepackage[utf8]{inputenc}
\usepackage[english,russian]{babel}
\usepackage{listings}
%\usepackage{columns}

\lstset{ language=Lisp,
         showspaces=false,
         morekeywords={define, begin, car, cdr}}

%
%Redefining sections as problems
%
\makeatletter
\newenvironment{problem}{\@startsection
       {section}
       {1}
       {-.2em}
       {-3.5ex plus -1ex minus -.2ex}
       {2.3ex plus .2ex}
       {\pagebreak[3]%forces pagebreak when space is small; use \eject for better results
       \large\bf\noindent{Задача }
       }
       }
       {%\vspace{1ex}\begin{center} \rule{0.3\linewidth}{.3pt}\end{center}}
       \begin{center}\large\bf \ldots\ldots\ldots\end{center}}
\makeatother


%
%Fancy-header package to modify header/page numbering 
%
\usepackage{fancyhdr}
\pagestyle{fancy}
%\addtolength{\headwidth}{\marginparsep} %these change header-rule width
%\addtolength{\headwidth}{\marginparwidth}
\lhead{Задача \thesection}
\chead{} 
\rhead{\thepage} 
\lfoot{\small\scshape MIPTScheme} 
\cfoot{} 
\rfoot{\footnotesize iLab МФТИ} 
\renewcommand{\headrulewidth}{.3pt} 
\renewcommand{\footrulewidth}{.3pt}
\setlength\voffset{-0.25in}
\setlength\textheight{648pt}

%%%%%%%%%%%%%%%%%%%%%%%%%%%%%%%%%%%%%%%%%%%%%%%

%
%Contents of problem set
%    
\begin{document}

\title{Обзор языка Scheme \\
      {\small  MIPT Scheme compiler }}
%\author{Арсений Заостровных}
\date{27.10.12}

\maketitle

\thispagestyle{empty}

\begin{problem}{\it Списки}
Дан список: '(a (b (c d) (e) ()) (f (g)))

Выпишите:
\begin{enumerate}
\item Его длину
\item Его же с помощью ``точечных пар'' -- (x . y)
\item Программу, получающую элемент 'e из этого списка, с помощью функций car и cdr.
\end{enumerate}
\end{problem}

\begin{problem}{\it Переменные}
Что напечатают следующие 2 программы:
\begin{lstlisting}
(define y 81)
(let ((x 1))
  (set! x 11)
  (let ((y (+ 1 x)) (x 18))
    (write (+ x y))))
  \end{lstlisting}
и
 \begin{lstlisting}
(let ((a (let ((b "b1")
               (c "c1"))
           c))
      (b "b2"))
  (let ((c a) (a b))
    (write a)
    (write b)))
  \end{lstlisting}
\end{problem}

\begin{problem}{\it Функции}
Дано определение функции:
\begin{lstlisting}
(define a 1)
(define (foo a b)
  (let ((b 10))
    (write a)
    (write b)
    (+ a b)))
\end{lstlisting}
Что напечатает следующая программа:
\begin{lstlisting}
(let ((x 100)
      (b (begin (write a) a)))
  (write (foo x b)))
\end{lstlisting}
?
\end{problem}

\begin{problem}{\it Рекурсия}
Напишите функцию (max lst) с использованием хвостовой рекурсии,
 выбирающую максимальный элемент из списка чисел.
\end{problem}

\begin{problem}{\it Замыкания}
Что напечатает следующая программа:
\begin{lstlisting}
(define (make-snake tail)
  (cons (lambda () tail)
        (lambda (x) (set! tail (cons tail x)))))

(let ((snake (make-snake '())))
  ((cdr snake) 42)
  (let ((python (make-snake ((car snake)))))
    ((cdr python) 84)
    (write ((car python)))
    (write ((car snake)))))
\end{lstlisting}
?
\end{problem}

\begin{problem}{\it Макросы}
Напишите макрос (push! a lst) добавляющий элемент в начало списка.
\end{problem}

\end{document}
