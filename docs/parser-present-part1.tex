\documentclass[16pt,pdf,unicode]{beamer}

\usepackage[utf8]{inputenc}
\usepackage[english,russian]{babel}
\usepackage{listings}
\usepackage{tikz}
\usepackage{color}
\usepackage{geometry}

\usetikzlibrary{positioning}
\usetikzlibrary{automata}
\usetikzlibrary{chains}
\usetikzlibrary{shapes,arrows}
\usetikzlibrary{calc} % for manimulation of coordinates
\tikzset{
    invisible/.style={opacity=0},
    visible on/.style={alt=#1{}{invisible}},
    alt/.code args={<#1>#2#3}{%
      \alt<#1>{\pgfkeysalso{#2}}{\pgfkeysalso{#3}} % \pgfkeysalso doesn't change the path
    },
  }

\usetheme{Boadilla}
\setbeamercovered{transparent}
%\lstset{ language=Lisp }

\title[Синтаксический анализ]{Теория и практика компилляции программ: синтаксический анализ, часть I}
\author[Д. Анисимов]{Денис Анисимов}
\institute[МФТИ]{Московский физико-технический университет\\
	{\tiny государственный университет}\\}
\date[17.11.2012]{МФТИ, 17 ноября 2012 года}

\begin{document}

\begin{frame}[plain]
\setcounter{framenumber}{0}
\titlepage
\end{frame}

\section{Введение}
\subsection{Обзор}
\begin{frame}
\frametitle{Положение синтаксического анализатора}
\begin{figure}
  \include{lexer_parser_role}
\end{figure}
\end{frame}

\subsection{Контекстно-свободные грамматики}
\begin{frame}
\frametitle{Контекстно-свободные грамматики(CFGs)}
\begin{itemize}
  \item Языки программирования имеют явную рекурсивную структуру
  \begin{itemize}
    \item CFG отражают эту структуру
  \end{itemize}
  \item CFG определяет элементарные элементы языка и правила получения составных элементов
  \begin{itemize}
    \item Набор терминальных символов $N$. В случае ЯП, терминал=токен
    \item Наборе нетерминальных(промежуточных) символов $N$
    \item Стартовый символ $S \in N$
    \item Набор правил(продукций) $P$:
      \begin{align*}
        X &\rightarrow Y_1...Y_n \\
        X &\in N \\
        Y_i &\in N \cup T \cup \{\varepsilon\}
      \end{align*}
  \end{itemize}
  \end{itemize}
\end{frame}

\begin{frame}
\frametitle{Пример CFG}
\begin{align*}
  stmt &\rightarrow \bold{if} \: expr \: \bold{then} \: stmt \: \bold{else} \: stmt\\
  expr &\rightarrow \bold{num} \: rel \: \bold{num}\\
  rel &\rightarrow \bold{>}\\
  rel &\rightarrow \bold{<}\\
  rel &\rightarrow \bold{==}
\end{align*}
\end{frame}

\begin{frame}
\frametitle{Порождаемый язык}
\begin{itemize}
  \item Строки языка \textit{порождаются} продукциями начиная со стартового символа и пока не останется нетерминалов:
  \begin{eqnarray*}
    &S \rightarrow ... \rightarrow ... \rightarrow \alpha_0 \rightarrow \alpha_1 \rightarrow ... \rightarrow \alpha_n \\
    &\alpha_0 \overset{*}{\rightarrow} \alpha_n
  \end{eqnarray*}
  \begin{itemize}
    \item Левое и правое порождение(left-, rightmost derivation)
  \end{itemize}
  \item Множество всех строк порождаемых грамматикой $G$ составляет контекстно-свободный язык $L(G)$:
  \begin{equation*}
    \{a_1...a_n\mid\forall i: a_i \in T, S\overset{*}{\rightarrow}a_1...a_n\}
  \end{equation*}
\end{itemize}
\end{frame}

\begin{frame}
\frametitle{Регулярные грамматики}
\begin{itemize}
  \item Регулярные грамматики являются подмножеством контекстно-свободных
    \begin{align*}
      S &\rightarrow a B \\
      C &\rightarrow \varepsilon
    \end{align*}
  \item Почему не делать лексический анализ вместе с синтаксическим?
    \begin{itemize}
      \item Лексические правила проще.
      \item Регулярные выражения понятны и легко реализуемы.
      \item Разделение на логические модули компилятора.
    \end{itemize}
\end{itemize}
\end{frame}

\subsection{Деревья разбора}
\begin{frame}
\frametitle{Дерево разбора}
\begin{itemize}
  \item Графическое представление порождения
  \item Корень дерева - стартовый символ, листья - терминальные символы.
  \item Пример для $id + id * id$:
    \begin{columns}
      \begin{column}{0.5\textwidth}
        \begin{center}
          \only<2-3>{\alert{\only<2>{Левое}\only<3>{Правое} порождение}}
        \end{center}
        \begin{align*}
        & E \rightarrow E \: A \: E \mid (E) \mid -E \mid \bold{id}\\
        & A \rightarrow + \mid - \mid * \mid /
        \end{align*}
      \end{column}
      \begin{column}{0.5\textwidth}
        \only<1-2>{
        \begin{figure}
        \begin{tikzpicture}
          \node {E}
            child {
              node {E}
                child {node {\bf{id}}}
            }
            child {node {\bf{+}}}
            child {
              node {E}
                child {
                  node {E}
                    child{node {\bf{id}}}
                }
                child {node {\bf{*}}}
                child {
                  node {E}
                    child{node {\bf{id}}}
                }
            }
          ;
        \end{tikzpicture}      
        \end{figure}
        }
        \only<3>{
        \begin{figure}
        \begin{tikzpicture}
          \node {E}
            child {
              node {E}
                child {
                  node {E}
                    child{node {\bf{id}}}
                }
                child {node {\bf{+}}}
                child {
                  node {E}
                    child{node {\bf{id}}}
                }
            }            
            child {node {\bf{*}}}
            child {
              node {E}
                child {node {\bf{id}}}
            }
          ;
        \end{tikzpicture}      
        \end{figure}
        }
      \end{column}
    \end{columns}
\end{itemize}
\end{frame}

\begin{frame}
\frametitle{Дерево абстрактного синтаксиса}
\begin{itemize}
  \only<1>{
  \item Не содержит информации не нужной для последующих фаз компилятора:
  \begin{itemize}
    \item Скобки
    \item Промежуточные узлы
    \item Управляющие символы
  \end{itemize}
  }
  \only<2>{
  \item Пример:\\
      \begin{figure}[t]
      \begin{tikzpicture}
        \node {stmt} 
          child {node {\bf{if}}}
          child {node {\bf{(}}}
          child {
            node {exp}
              child {node {\bf{0}}}
          }
          child {node {\bf{)}}}
          child {node {\bf{then}}}
          child {
            node {stmt}
              child {node{\bf{other}}}
          }
          child {node {\bf{else}}}
          child {
            node {stmt}
              child {node{\bf{other}}}
          }
        ;
      \end{tikzpicture}
      \end{figure}
      \begin{figure}[b]
      \begin{tikzpicture}
        \node {stmt}
          child {node {\bf{0}}}
          child {node {\bf{other}}}
          child {node {\bf{other}}}
        ;
      \end{tikzpicture}
      \end{figure}
  }
\end{itemize}
\end{frame}

\subsection{Неоднозначности}
\begin{frame}
\frametitle{Неоднозначности}
\begin{itemize}
  \item Существуют различные порождения одной строки
  \item Отражают различную семантику операций
  \item Пример для $id + id * id$:
    \begin{columns}
      \begin{column}{0.5\textwidth}
        \begin{figure}
        \begin{tikzpicture}
          \node {E}
            child {
              node {E}
                child {node {\bf{id}}}
            }
            child {node {\bf{+}}}
            child {
              node {E}
                child {
                  node {E}
                    child{node {\bf{id}}}
                }
                child {node {\bf{*}}}
                child {
                  node {E}
                    child{node {\bf{id}}}
                }
            }
          ;
        \end{tikzpicture}      
        \end{figure}
      \end{column}
      \begin{column}{0.5\textwidth}
        \begin{figure}
        \begin{tikzpicture}
          \node {E}
            child {
              node {E}
                child {
                  node {E}
                    child{node {\bf{id}}}
                }
                child {node {\bf{+}}}
                child {
                  node {E}
                    child{node {\bf{id}}}
                }
            }            
            child {node {\bf{*}}}
            child {
              node {E}
                child {node {\bf{id}}}
            }
          ;
        \end{tikzpicture}      
        \end{figure}
      \end{column}
    \end{columns}
\end{itemize}
\end{frame}

\begin{frame}
\frametitle{Устранение неоднозначностей}
\begin{itemize}
  \item Изменение грамматики
  \item Использование правил устранения неоднозначности
\end{itemize}
\end{frame}

\begin{frame}
\frametitle{Устранение неоднозначности из грамматики}
\begin{itemize}
  \item Рассмотрим грамматику:
  $$ S \rightarrow \bold{if} \: E \: \bold{then} \: S \: \bold{else} \: S $$
  \item ... и строку
  $$ \bold{if} \: E_1 \: \bold{then} \: \bold{if} \: E_2 \: \bold{then} \: S_1 \: \bold{else} \: S_2 $$
  \item \textbf{Else} должен соответствовать ближайшему \textbf{then}:
  \begin{align*}
    & S \rightarrow M \mid U \\ 
    & M \rightarrow \bold{if} \: E \: \bold{then} \: M \: \bold{else} \: M \\
    & U \rightarrow \bold{if} \: E \: \bold{then} \: S \mid 
        \bold{if} \: E \: \bold{then} \: M \: \bold{else} \: U
  \end{align*}
\end{itemize}
\end{frame}

\begin{frame}
\frametitle{Устранение неоднозначности из грамматики}
\begin{itemize}
  \item Можно оставить неоднозначность грамматики, но указать парсеру семантические особенности языка:
    \begin{itemize}
      \item Ассоциативность операций(\%left)
      \item Приоритет операций (\%left '+' \%left '*')
    \end{itemize}
\end{itemize}
\end{frame}

\subsection{Обработка ошибок}
\begin{frame}
\frametitle{Обработка ошибок}
\begin{itemize}
  \item Цель компилятора - транслировать правильные программы и отслеживать ошибочные
  \item Способы обработки ошибок:
  \begin{itemize}
    \item Режим паники \\ Пропустить всё до следующего синхронизирующего токена
    \item Продукции ошибок \\ Записать в грамматике наиболее частые ошибочные конструкции
    \item Коррекция \\ Попытаться изменить ошибочную программу до ближайшей рабочей
  \end{itemize}
  \item В современных распротранённых компиляторах сложные обработчики ошибок не применяются
\end{itemize}
\end{frame}

\section{Нисходящий анализ}

\begin{frame}[plain]
  \begin{center}
    \structure{{\Large Нисходящий анализ}\\
      \vspace{1cm}
      {\large Метод рекурсивного спуска}}
  \end{center}
\end{frame}

\addtocounter{framenumber}{-1}

\subsection{Метод}
\begin{frame}
\frametitle{Построение дерева разбора}
\begin{itemize}
  \item Дерево разбора строится сверху-вниз слева-направо.
  \item Терминальные символы рассматриваются в порядке поступления.
\end{itemize}
\end{frame}

\begin{frame}
\frametitle{Пример}
\begin{itemize}
  \item Рассмотрим грамматику:
  \begin{align*}
    E &\rightarrow T \: \mid \: T \: + \: E \\
    T &\rightarrow \bold{int} \mid \bold{int} \: * \: T \: \mid (E)
  \end{align*}
  \item Входная строка
    $$(\bold{int}_5)$$
  \item Начинаем со стартового нетерминала $E$ и перебираем продукции в порядке их объявления
\end{itemize}
\end{frame}

\begin{frame}
\frametitle{Пример}
\begin{columns}
  \begin{column}{0.5\textwidth}
  \begin{align*}
    E &\rightarrow \alert<2,9>{T} \: \mid \: T \: + \: E \\
    T &\rightarrow \alert<3,10>{\bold{int}} \mid \alert<5>{\bold{int}} \: \alert<5>{*} \: \alert<5>{T} \: \mid \alert<7>{(E)}
  \end{align*}
  \vspace{1cm}  
  \begin{align*}
    \alert<1-7,13>{(} & \alert<8-10,13>{\bold{int}_5} \alert<11,13>{)} \\
    &\only<8,11,12>{\alert{\rightarrow}}
  \end{align*}
  \end{column}
  \begin{column}{0.5\textwidth}
  \begin{figure}
  \begin{tikzpicture}
    \node {E}
      child [visible on=<2->] {
        node {T} 
          child [visible on=<3>] {node {\alert{\bf{int}}}}
        node {}
          child [visible on=<5>] {node {\alert{\bf{int}}}}
          child [visible on=<5>] {node {\alert{*}}}
          child [visible on=<5>] {node {\alert{T}}}
        node {}
          child [visible on=<7->] {node {\alert<13>{(}}}
          child [visible on=<7->] {
            node {E}
              child [visible on=<9->] {
                node {T}
                  child [visible on=<10->] {node {\alert<13>{\bf{int}}}}
              }
          }
          child [visible on=<7->] {node {\alert<13>{)}}}
      }
    ;
  \end{tikzpicture}
  \end{figure}
  \end{column}
\end{columns}
\end{frame}

\subsection{Левая рекурсия}
\begin{frame}
\frametitle{Левая рекурсия}
\begin{itemize}
  \item Рассмотрим продукцию:
    $$ S \rightarrow Sa $$
  \item ... и метод рекурсивного спуска входит в бесконечный цикл
  \item Грамматика называется леворекурсивной, если в неё входит продукция вида:
    $$ S \rightarrow^+ S\alpha $$
  \item Такие грамматики не подходят для метода рекурсивного спуска
\end{itemize}
\end{frame}

\begin{frame}
\frametitle{Пример устранения левой рекурсии}
\begin{itemize}
  \item Рассмотрим леворекурсивную грамматику:
    $$ S \rightarrow S\alpha \: \mid \: \beta  $$ 
  \item $S$ порождает все строки начинающиеся с $\beta$ с произвольным числом последующих $\alpha$.
    $$ S \rightarrow S\alpha \: \rightarrow S\alpha\alpha \: \rightarrow \dots \rightarrow S\alpha\dots\alpha \rightarrow \beta\alpha\dots\alpha $$ 
  \item Можно переписать использую правую рекурсию:
    \begin{align*}
      &S \rightarrow \beta S' \\
      &S' \rightarrow \alpha S' \: \mid \: \varepsilon
    \end{align*}
  
\end{itemize}
\end{frame}

\begin{frame}
\frametitle{Левая факторизация}
\begin{itemize}
  \item Рассмотрим грамматику:
    $$ S \rightarrow \alpha\beta_1 \: \mid \: \alpha\beta_2  $$ 
  \item Она не совместима с методом рекурсивного спуска с возвратом - однажды поглотив символ $\alpha$ с использованием первой продукции, он не сможет использовать вторую.
  \item Устранение этой неоднозначности - левая факторизация:
    \begin{align*}
      &S \rightarrow \alpha S' \\
      &S' \rightarrow \beta_1 \: \mid \: \beta_2
    \end{align*}
  
\end{itemize}
\end{frame}

\end{document}


