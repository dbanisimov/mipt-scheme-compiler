\documentclass[a4,12pt]{article}

%AMS-TeX packages
\usepackage{amssymb,amsmath,amsthm} 
%geometry (sets margin) and other useful packages
\usepackage[margin=1in]{geometry}
\usepackage{graphicx,ctable,booktabs}
\usepackage[utf8]{inputenc}
\usepackage[english,russian]{babel}
\usepackage{tikz}
\usepackage{color}
\usepackage{geometry}
\usepackage[pdftex,
            pdfauthor={Denis Anisimov},
            pdftitle={Parser quiz},
            pdfsubject={MIPT-Scheme Compiler},
            pdfkeywords={mipt, scheme, compiler}]{hyperref}

%%%%%%%%%%%%%%%%%%%%%%%%%%%%%%%%%%%%%%%%%%%%%%%

%
%Contents of problem set
%    
\begin{document}

\title{Синтаксический анализ. \\
      {\small  MIPT Scheme compiler }}
\date{15.12.12}

\maketitle

\thispagestyle{empty}

Рассмотрим грамматику:
\begin{align*}
  S &\rightarrow (L) \: \mid \: a \\
  L &\rightarrow L,S \: \mid \: S
\end{align*}

\begin{enumerate}
  \item (1) Что в ней является терминалами, нетерминалами, стартовым символом?
  \item (1) Постройте деревья разбора, левое и правое порождение для следующих предложений:
    \begin{enumerate}
      \item $(a,a)$
      \item $(a,(a,a))$
      \item $(a,((a,a),(a,a)))$
    \end{enumerate}
  \item (1) Устраните \textit{все} неоднозначности из данной грамматики.
  \item (2) Напишите последовательность порождений для анализатора, работающего по методу рекурсивного спуска с откатом, для строки $(a,((a),a))$
  \item (3) Постройте множества FIRST и FOLLOW.
  \item (4) Постройте таблицу разбора предиктивного анализатора.
  \item (1) Постройте правое порождение для $(a,(a,a))$ и покажите основы каждой правосентенциальной формы.
  \item (1) Покажите шаги ПС-анализатора и соответствующее построение дерева разбора для порождения из \textbf{7.}
  \item (1) Постройте замыкание для пункта $S \rightarrow (\cdot L)$
  \item (1) Постройте операцию $goto(I,L)$, где $I$ - замыкание из \textbf{9.}
  \item (3) Постройте каноническую систему множеств LR(0)-пунктов.
  \item (3) Постройте ДКА, соответствующий переходам $goto$ для канонической системы LR(0)-пунктов.
  \item (4) Постройте таблицу действий и таблицу переходов SLR-анализатора.
\end{enumerate}

\end{document}
